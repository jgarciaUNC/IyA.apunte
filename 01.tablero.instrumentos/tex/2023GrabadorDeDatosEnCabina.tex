
\subsection{Registrador de datos de vuelo}
\label{sec:registrador.datos.vuelo}

tanto de comportamiento de la aeronave como de alguno de sus componentes, o en caso de algún incidente o accidente. Este equipo de grabación se coloca dentro de una caja blindada y resistente a golpes y fuego y se instala generalmente hacia la cola en la parte posterior del fuselaje. Los elementos sobre los que queda grabada la información también son resistentes al fuego, de forma que aun en caso de algún accidente la caja se puede recuperar con la información que contiene que siempre es de gran valor. * McDouglas   En esta figura se presenta una caja blindada y el equipo que va dentro, junto con el panel de control desde donde se maneja y se le introducen los datos de fecha y número de vuelo. Existen distintos sistemas de grabación, uno con agujas sobre cinta metálica, de forma gráfica, ya muy antigua y poco usada, la cual graba pocos parámetros; y el más usado y que graba más parámetros, es digital sobre cinta magnetofónica. Los datos que generalmente se graban son los necesarios para que una vez finalizado un vuelo, se pueda reconstruir si la situación lo requiere. Además de los 
varios parámetros de motor, etc., los fabricantes también programan la grabación de los parámetros que considere la compañia operadora y los que las reglamentaciones de cada pais obligan para un mejor estudio de las condiciones de vuelo. Hay un panel de control en el que se selecciona el numero de vuelo y fecha. En él también se encuentra un indicador que tras ser pulsado el botón de Test y estando el equipo en buen estado la aguja indica el arco verde. También hay un botón con la literatura de event, que sirve para remarcar algún momento critico del vuelo por decisión de la tripulación.

En los FDR de cinta metalica la toma de datos de aire era introducida directamente por medio de tuberias desde la toma dinamica y estatica. Sin embargo, en los nuevos FDR los datos de aire son enviados por los ADC (Calculadores de Datos de Aire). En los nuevos sistemas, el control dia, hora y vuelo, son suministrados automaticamente por el sistema, al introducir los datos desde la MCDU. Un interruptor permite activar el sistema completo estando la aeronave en tierra, pudiendo realizarse de esta forma pruebas a los circuitos antes del vuelo. El sistema de Test va independiente y solo debe ser actuado en tierra, por ello generalmente esta bajo guarda. Asi mismo lleva una luz de aviso OFF para el caso en que el FDRS esté inoperativo. Fijado al frontal de la caja registradora, se coloca una baliza ULB (Underwater Locator Beacon) que emitirá señal de situación para que pueda ser localizada por los equipos de rescate si por cualquier causa la aeronave termina en el agua.




Gato Gutiérrez, F. (2013). Sistemas de aeronaves de turbina. Tomo I. San Vicente(Alicante), Spain: ECU. Recuperado de https://elibro.net/es/ereader/bmayorunc/62295?page=384.






\subsection{Grabador de voces en cabina}
\label{sec:grabador.voces.en.cabina}

En la actualidad en las aeronaves comerciales se instala un equipo de grabación de voces en la cabina de pilotos con el objetivo de complementar los datos grabados por el equipo de parámetros de vuelo. Se compone de: panel de control, micrófonos y equipo grabador, para grabar todo lo emitido por las ondas de control de audio: VHF y HF, además de todo lo que se habla en la cabina de mandos. El panel de control y los micrófonos se sitúan en la cabina, en el panel de control lleva un micrófono, un botón de borrado, otro de test, asi como un conector hembra para enchufar unos microteléfonos, si se han instalado dos micrófonos uno irá colocado en el frontal de instrumentos.




Gato Gutiérrez, F. (2013). Sistemas de aeronaves de turbina. Tomo I. San Vicente(Alicante), Spain: ECU. Recuperado de https://elibro.net/es/ereader/bmayorunc/62295?page=386.


El equipo grabador esta situado dentro de una caja similar a la registradora de datos, y colocadas las dos en la misma zona de la aeronave. En el interior esta el equipo grabador con una cinta que no suele tener mas de treinta minutos de duración, es decir, que como es reversible vuelve a grabar sobre lo anterior, tiempo suficiente para complementar cualquier investigación o aclaración posterior a un vuelo.

Gato Gutiérrez, F. (2013). Sistemas de aeronaves de turbina. Tomo I. San Vicente(Alicante), Spain: ECU. Recuperado de https://elibro.net/es/ereader/bmayorunc/62295?page=387.

En esta figura se presenta un esquema completo del sistema grabador de voces en cabina donde pueden observarse las funciones, la forma de ejecutarlas y las señales de entrada y salida, asi como también que para borrar es necesario tener puestos los frenos de aparcamiento, con la aeronave en tierra.


Gato Gutiérrez, F. (2013). Sistemas de aeronaves de turbina. Tomo I. San Vicente(Alicante), Spain: ECU. Recuperado de https://elibro.net/es/ereader/bmayorunc/62295?page=387.