%****** Acronyms and abbreviations in avionics ******
%***** [edit] A *****

% \acrodef{AGPL}{Affero General Public License}
% AGPL: Affero General Public License.




%     * ACARS: Aircraft Communications Addressing and Reporting System
%     * ACAS: Airborne Collision Avoidance System
%     * ACP: Audio control panel
%     * ACS: Audio control system
%     * ADAHRS: Air data and attitude heading reference system
%     * ADC: Air data computer
%     * ABAS: Aircraft-based augmentation system
%     * ADF: Automatic direction finder
%     * ADI: Attitude director indicator
%     * ADIRS: Air Data Inertial Reference System
%     * ADIRU: Air data inertial reference unit
%     * ADM: Air data module
%     * ADS: Either; Automatic Dependent Surveillance or air data system
%     * ADS-A: Automatic Dependent Surveillance-Address
%     * ADS-B: Automatic Dependent Surveillance-Broadcast
%     * AFCS: Automatic flight control system
%     * AFD: Autopilot flight director
%     * AFDC: Autopilot flight director computer
%     * AFDS: Autopilot flight director system
%     * AFIS: Either; Automatic flight information service or airborne flight
%       information system
%     * AGACS: Automatic ground–air communications system, is also known as ATCSS
%       or data link
%     * AGDL: Air-Ground Data Link
%     * AGC: Automatic gain control
%     * AHC: Attitude heading control
%     * AHRS: Attitude and Heading Reference Systems

\acrodef{AIDS}{Aircraft Integrated Data System}
      AIDS: Aircraft Integrated Data System

%     * ALC: Automatic level control
%     * ALT: Either; altimeter or altitude
%     * ALT Hold: altitude hold mode
%     * ALTS: Altitude select
%     * AMLCD: Active-matrix liquid crystal display
%     * ANC: Active noise cancellation
%     * ANN: Annunciator – caution warning system normally containing visual and
%       audio alerts to the pilot
%     * ANR: Active noise reduction
%     * ANT: Antenna
%     * A/P: Autopilot
%     * APC: Autopilot computer
%     * APS: Autopilot system
%     * APU: Auxiliary power unit
\acrodef{ARINC}{Aeronautical Radio INCorporated}
	ARINC: Aeronautical Radio INCorporated 

%     * ASD: Aircraft situation display
%     * ASDL: Aeronautical satellite data link
%     * ASR: Airport surveillance radar
%     * ASU: Avionics switching unit
\acrodef{ATC}{Air traffic control} ATC: Air traffic control
%     * ATCRBS: Air Traffic Control Radar Beacon System
%     * ATCSS: Air traffic control signaling system
%     * ATI: Unit of measure for instrument size, a standard 3¨û cutout is a 3ATI
%     * ATM: Air traffic management

\acrodef{ATSU}{Air Traffic Services Unit}
	ATSU: Air Traffic Services Unit%, implementada por AIRBUS en sus aviones (no en los A300/310), 
%		realiza las funciones generalmente usuales en unidades de gesti\'on de comunicaciones
%		de otro tipo de aviones.

%     * Avionics: Aviation + electronics
%     * AWG: American wire gauge


% ***** [edit] B *****
%     * B RNAV: Basic area navigation
%     * BARO: Barometric indication, setting or pressure
%     * BCRS: Back course
%     * BDI: Bearing distance indicator
%     * BGAN: Broadcast Global Area Network
\acrodef{BIT}{BInary Digit}
	BIT: BInary Digit

% ***** [edit] C *****
%     * CAI: Caution annunciator indicator
%     * CAT I: Operational performance Category 1
%     * CAT I Enhanced Allows for lower minimums than CAT I in some cases to CAT
%       2 minimums
%     * CAT II: Operational performance Category II
%     * CAT IIIa: Operational performance Category IIIa
%     * CAT IIIb: Operational performance Category IIIb
%     * CAT IIIc: Operational performance Category IIIc
%     * CODEC: Coder/decoder
%     * CDI: Course deviation indicator
%     * CDTI: Cockpit display of traffic information

\acrodef{CFDS}{Centralised Fault Display System}
	CFDS: Centralised Fault Display System

%     * CFIT: Controlled flight into terrain

\acrodef{CDU}{Control Display Unit}
	CDU: Control Display Unit

%     * COMM or COM: Communications receiver
%     * CNS: Communication, navigation, surveillance
%     * CNS/ATM: Communication, navigation, surveillance/air traffic Management
%       [1]
%     * CPDLC: Controller–pilot data link communications
%     * CPS: Cycles per second
%     * CRT: Cathode ray tube
\acrodef{CRT}{Cathode Ray Tube}
	CRT: Cathode Ray Tube
%     * CTAF: Common traffic advisory frequency
%     * CV/DFDR: Cockpit voice and digital flight data recorder
%     * CVR: Cockpit voice recorder
%     * CWS: Control wheel steering
% ***** [edit] D *****
%     * DA: Drift angle
%     * DAPs: Downlink of aircraft parameters
%     * DCDU: Data link control and display unit
%     * DG: Directional gyroscope
%     * DGPS: Differential global positioning system
%     * DH: Decision height
\acrodef{DITS}{Digital Information Transfer System}
	DITS: Digital Information Transfer System
%     * DLR: Data link recorder
%     * DME: Distance measuring equipment
%     * DNC: Direct noise canceling
%     * DP: Departure procedures
%     * DSP: Digital signal processing
%     * DUAT: Direct user access terminal
% ***** [edit] E *****
%     * EADI: Electronic attitude director indicator

\acrodef{EFD}{Electronic flight display}
      EFD: Electronic flight display

%     * EFIS: Electronic flight instrument system
%     * EGPWS: Enhanced ground proximity warning system
%     * EGT: Exhaust gas temperature
%     * EHS: Enhanced surveillance
%     * EHSI: Electronic horizontal situation indicator

\acrodef{EICAS}{Engine indication crew alerting system}
      EICAS: Engine indication crew alerting system

%     * ELT: Emergency locator transmitter
%     * ENC: Electronic noise canceling
%     * ENG: Engine
%     * ENR: Electronic noise reduction
%     * EPR: Engine pressure ratio
%     * ETOP: Extended-range twin-engine operation
% ***** [edit] F *****
%     * FADEC: Full authority digital engine control
%     * FANS: Future Air Navigation System
%     * FAT: Free air temperature
%     * FDPS: Flight plan Data Processing System
%     * FDRS: Flight data recorder system
%     * FDU: Flux detector unit
%     * FF: Fuel flow
%     * FIS-B: Flight information services – broadcast
%     * FLIR: Forward-looking infra-red
%     * FLTA: Forward-looking terrain avoidance

\acrodef{FMC}{Flight Management Computer}
	FMC: Flight Management Computer

\acrodef{FMGS}{Flight Management Guidance System}
	FMGS: Flight Management Guidance System

\acrodef{FMS}{Flight Management System}
      FMS: Flight Management System

%     * FREQ: Frequency
%     * FSS: Flight service station
%     * FWS: Flight warning system
%     * FYDS: Flight director/ Yaw damper system

% ***** [edit] G *****

%     * GBAS: Ground based augmentation system
%     * GCAS: Ground collision avoidance system
%     * GCU: Generator control unit
%     * GDOP: Geometric dilution of precision
%     * GGS: Global positioning system ground station
%     * GHz: Gigahertz
%     * GLNS: GPS Landing and Navigation System
%     * GLNU: GPS landing and navigation unit
%     * GLONASS: Global Navigation Satellite System
%     * GLS: GPS Landing System
%     * GLU: GPS landing unit

\acrodef{GND}{Ground}
	GND: Ground (tierra)


    % * GNSS: Global Navigation Satellite System
    % * GMT: Greenwich Mean Time

\acrodef{GPS}{Global Positioning Satellite or Global Positioning System}
    GPS: Global Positioning Satellite or Global Positioning System

    % * GPWC: Ground proximity warning computer

\acrodef{GPWS}{Ground proximity warning system}
     GPWS: Ground proximity warning system

% ***** [edit] H *****
%     * HDG: Heading
%     * HDG SEL: Heading select
%     * HDOP: Horizontal dilution of precision
%     * HF: High frequency
%     * HHLD: Heading hold
%     * HSD: High-speed data
%     * HSI: Horizontal situation indicator
%     * HSL: Heading select
%     * HUD: Head-up display
%     * HMD: Helmet-mounted display
% ***** [edit] I *****
%     * IAS: Indicated airspeed
%     * ID: Identify/Identification or identifier
%     * IDENT: Identify/identifier
%     * IDS: Information display system or integrated display system
%     * IFE: In-flight entertainment
%     * IFR: Instrument flight regulations
%     * ILS: Instrument landing system
%     * IMC: Instrument meteorological conditions
%     * InHg: Inch of Mercury
%     * IND: Indicator[disambiguation needed]
%     * INS: Inertial Navigation System
%     * ISA: International Standard Atmosphere
%     * ISP: Integrated switching panel
%     * ITT: Interstage turbine temperature
%     * IVSI: Instantaneous vertical speed indicator
% ***** [edit] J *****
%     * JTIDS: Joint Tactical Information Distribution System
% ***** [edit] L *****
%     * LAAS: Local Area Augmentation System
%     * LADGPS: Local Area Differential GPS
%     * LCD: Liquid crystal display
\acrodef{LCD}{Liquid Crystal Display}
	LCD: Liquid Crystal Display

%     * LDGPS: Local area differential global positioning satellite
%     * LED: Light-emitting diode
%     * LMM: Locator middle marker
%     * LOC: Localizer
%     * LOM: Locator outer marker
%     * LORAN: Long-range navigation
%     * LRU: Line-replaceable unit
% ***** [edit] M *****
%     * MAP: Manifold absolute pressure or missed approach point
%     * MB: Marker beacon
%     * MCBF: Mean cycles between failures

\acrodef{MCDU}{Multi Control Display Unit}
	MCDU: Multi Control Display Unit

%     * MDA: Minimum decent altitude
%     * MEL: Minimum equipment list
%     * MF: Medium frequency
%     * MFD: Multi-function display
%     * MFDS: Multi-function display system
%     * MIC: Microphone
%     * MIDS: Multifunctional information distribution system
%     * MILSPEC: Military specification
%     * MKR: Marker beacon
%     * MLS: Microwave landing system
%     * MM: Middle marker
%     * MNPS: [Minimmum navigation performance specifications]
%     * MMD: Moving map display
%     * MOA: Military operations area
%     * Mode A: Transponder pulse-code reporting
%     * Mode C: Transponder code and altitude reporting
%     * Mode S: Transponder code, altitude, and TCAS reporting
%     * MOSArt: Modular Open System Architecture
%     * MSG: Message
%     * MSP: Modes S-Specific Protocol
%     * MSSS: Mode S-Specific Services
%     * MTBF: Mean time between failures
%     * MTTF: Mean time to failure


%     * MVFR: Marginal visual flight rules
% ***** [edit] N *****
%     * NAS: National Airspace System
%     * NAV: Navigation receiver
%     * Navaid: Navigational aid
%     * NAVCOMM: Navigation and communications equipment or receiver
%     * NAVSTAR-GPS: The formal name for the space-borne or satellite navigation
%       system
%     * NCATT: National Center for Aircraft Technician Training
%     * ND: Navigation display
%     * NDB: Non-directional radio beacon
%     * NFF: No fault found
%     * NM or NMI: Nautical mile
%     * NoTAM: Notice to airmen
%     * NPA: Non-precision approach
%     * NVD: Night vision device
%     * NVG: Night vision goggles

% ***** [edit] O *****
%     * OAT: Outside air temperature
%     * OBS: Omnibearing selector
%     * OM: Outer marker
\acrodef{OLED}{Organic Light-Emitting Diode}
	OLED: Organic Light-Emitting Diode

% ***** [edit] P *****
%     * PA: Public address system
%     * P-Code: GPS precision code
%     * PAPI: Precision approach path indicator
%     * PAR: Precision approach radar
%     * PD: Profile descent
%     * PDOP: Position dilution of precision
%     * PFD: Primary flight display or primary flight director
%     * PMG: Permanent magnet generator
%     * PND: Primary navigation display
%     * PNR: Passive noise reduction
%     * POS: Position[disambiguation needed]
%     * P-RNAV: precision area navigation
%     * PSR: Primary surveillance radar
%     * PTT: Push-to-talk
% ***** [edit] R *****
%     * RA: Resolution advisory (TCAS)
%     * RAI: Radio altimeter indicator
%     * RAIM: Receiver-autonomous integrity monitoring, also remote autonomous
%       integrity monitoring
%     * RALT: Radar or radio altimeter
%     * RAT: Ram air turbine
%     * RCR: Reverse current relay
%     * RCVR: Receiver
%     * RDMI: Radio distance magnetic indicator
%     * RDP: Radar data processing system
%     * RDR: Radar
%     * REF: Reference
%     * REIL: Runway end identifier lights
%     * REL: Relative[disambiguation needed]
%     * RF: Radio frequency
%     * RFI: Radio frequency interference
%     * RHSM: Reduced horizontal separation minimal
%     * RLG: Ring laser gyroscope
%     * RLY: Relay
%     * RMI: Radio magnetic indicator
%     * R-NAV: Area navigation
%     * RNG: Range
%     * RNP: Required navigation performance
%     * ROC: Rate of climb
%     * ROD: Rate of descent
%     * RPA: Remotely piloted aircraft (Unmanned aerial vehicle)
%     * RPM: Revolutions per minute
%     * RTE: Route
%     * RVR: Runway visual range
%     * RVSM: Reduced vertical separation minimum
%     * RX: Receiver
% ***** [edit] S *****
%     * SAT: Static air temperature
%     * SATCOM: Satellite communication
%     * SATNAV: Satellite navigation
%     * SD: Secure digital
%     * SELCAL: Selective calling

\acrodef{SID}{Standard Instrument Departure}
      SID: Standard Instrument Departure

%     * SIU: Satellite interface unit
%     * S: Sensitivity Level
%     * SMS: Short Messaging Service
%     * SNR: Signal-to-noise ratio
%     * SPKR: Speaker
%     * SQ or SQL: Squelch
%     * SSCV/DR: Solid-state cockpit voice/data recorder
%     * SSCVR: Solid-state cockpit voice recorder
%     * SSFDR: Solid-state flight data recorder
%     * SSR: Secondary surveillance radar

\acrodef{STAR}{Standard Terminal Arrival Route}
      STAR: Standard Terminal Arrival Route

\acrodef{STARS}{Standard Terminal Automation Replacement System}
     STARS: Standard Terminal Automation Replacement System



%     * STC: Supplemental Type Certificate
%     * STCA: Short-Term Conflict Alert
%     * STP: Standard temperature and pressure
%     * SUA: Special use airspace
% ***** [edit] T *****
%     * TA: Traffic advisory (see TCAS)
%     * TACAN: Tactical air navigation system
%     * Tach: Tachometer
%     * TAD: Terrain awareness display
%     * TAF: Terminal area forecast
%     * TAS: True airspeed
%     * TAT: True air temperature, or total air temperature
%     * TAWS: Terrain awareness warning system
%     * TBO: Time before overhaul, or time between overhaul
%     * TCA: Throttle control assembly, or terminal control area
%     * TCAS: Traffic collision avoidance system
%     * TCF: Terrain clearance floor
%     * TCN: TACAN
%     * TCU: TACAN control unit
%     * TDOP: Time dilution of precision
%     * TDR: Transponder (in some cases)
%     * TERPS]]: Terminal instrument procedures, or terminal enroute procedures
%     * TFR: Temporary flight restrictions
%     * TFT: Thin-film transistor
%     * TGT: Turbine gas temperature, or target
%     * THDG: True heading
%     * TIAS: True indicated airspeed
%     * TIS: Traffic information service
%     * TK: Track angle
%     * TKE: Track-angle error
%     * TLA: Three-letter acronym
%     * TOGA: Takeoff/Go-around switch, Takeoff/go-around thrust
%     * TOT: Turbine outlet temperature
%     * TR or T/R: Transmitter receiver or transceiver
%     * TRACON: Terminal radar approach control
%     * TRANS: Transmit, Transmission, or Transition[disambiguation needed]
%     * TRK: Track
%     * TRP: Mode S transponder
%     * TTR: TCAS II transmitter/receiver
%     * TTS: Time to station
%     * TVE: Total vertical error
%     * TWDL: Two-way data link, or terminal weather data link
%     * TWDR: Terminal Doppler Weather Radar
%     * TWIP: Terminal weather information for pilots
%     * TWR: Terminal weather radar
%     * TX: Transmit
% ***** [edit] U *****
%     * UART: Universal asynchronous receiver transmitter
%     * UAV: Unmanned aerial vehicle
%     * UHF: Ultra-high frequency
%     * ULB: Underwater locator beacon
%     * USB: Universal Serial Bus
%     * UTC: Universal Time Coordinate
% ***** [edit] V *****
%     * V: Volts or voltage
%     * VASI: Visual approach slope indicator
%     * VDL: VHF Data Link
%     * VDR: VHF digital radio
%     * VFO: Variable frequency oscillator
%     * VFR: Visual flight rules
%     * VG/DG: Vertical gyroscope/directional gyroscope
%     * VGA: Video Graphics Array
%     * VHF: Very high frequency
%     * V/L: VOR/Localizer
%     * VMC: Visual meteorological conditions or minimum control speed with
%       critical engine out
%     * V/NAV: Vertical navigation
%     * VNE: Never exceed speed
%     * VNO: Maximum structural cruising speed
%     * VNR: VHF navigation receiver
%     * VOR: VHF omnidirectional range and ranging
%     * VOR/DME: VOR with Distance measuring equipment
%     * VOR/MB: VOR marker beacon
%     * VORTAC: VOR and TACAN combination
%     * VOX: Voice transmission
%     * VPATH: Vertical path
%     * V/R: Voltage regulator
%     * V/REF: Reference velocity
%     * V/S: Vertical speed
%     * VSI: Vertical speed indicator
%     * VSM: Vertical separation limit
%     * VSO: Stall speed in landing configuration
%     * VSWR: Voltage–standing wave ratio
%     * V/TRK: Vertical track
%     * VX: Speed for best angle of climb
%     * VY: Speed for best rate of climb
% ***** [edit] W *****
%     * WAAS: Wide Area Augmentation System
%     * WD/WINDR: Wind direction
%     * WMA: WXR waveguide adapter
%     * WMI: WXR indicator mount
%     * WMS: Wide-area master station
%     * WMSC: Weather message switching center
%     * WMSCR: Weather message switching center replacement
%     * WPT: Waypoint
%     * WRT: WXR receiver transmitter
%     * WX: Weather
%     * WXR: Weather radar system
%     * WYPT: Waypoint
% ***** [edit] X *****
%     * XCVR: Transceiver
%     * XFR: Transfer
%     * XMIT: Transmit
%     * XMSN: Transmission
%     * XMTR: Transmitter
%     * XPDR: Transponder
%     * XTK: Crosstrack
% ***** [edit] Y *****
%     * YD: Yaw damper
% ***** [edit] See also *****
%     * Avionics
%     * List of aviation, aerospace and aeronautical abbreviations
% ***** [edit] References *****
%    1. ^ http://wwwicaoint/icao/en/ro/rio/execsumpdf


% Retrieved from "http://en.wikipedia.org/w/
% index.php?title=Acronyms and abbreviations in avionics&amp;oldid=518531528"
