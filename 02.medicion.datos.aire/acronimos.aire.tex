\acrodef{QFE}{}
QFE : C\'odigo Q para presi\'on atmosf\'erica en la elevaci\'on del aer\'odromo, las letras representan \textbf{Field Elevation} (elevacion del campo o terreno). 
Si se fija el altímetro con la presión QFE que nos dé un aeródromo, este marcará 0 al despegar o aterrizar en el mismo.

\acrodef{QNH}{}
QNH: C\'odigo Q para presi\'on atmosf\'erica respecto al nivel del mar deducida de la existente en el aeródromo, considerando la atmósfera con unas condiciones estándar, es decir sin tener en cuenta las desviaciones de la temperatura real con respecto a la estándar. 

\acrodef{QNE}{}
QNE: C\'odigo Q para presi\'on altitud seg\'un atm\'osfera estandard internacional, 1013,25 hPa o 29,92 pulgadas de mercurio,  es la correspondiente a la atmósfera tipo al nivel del mar.  
Los reglamentos aéreos establecen que todos los aviones vuelen con la misma presión de referencia, de esta manera, cualquier cambio en las condiciones atmosféricas afectan por igual a todos los aviones, garantizando la altura de seguridad que los separa.
