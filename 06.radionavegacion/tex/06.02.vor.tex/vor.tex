\section*{VOR}

El denominado Radiofaro VHF Omnidireccional, es la ayuda a la navegación de corto alcance reconocida internacionalmente. Normalmente empleado para delinear aerovías en colaboración con el DME y en los aeropuertos pequeños se lo emplea en versión de baja potencia para como ayuda a la aproximación. Desarrollado en Estados Unidos, tal como se indicó anteriormente, a partir del radiofaro giratorio de los años veinte, se reconoció como estándar internacional en 1949 y, por lo tanto, ha reemplazado al radiofaro internacional de MF.

El principio de funcionamiento consiste en dos modulaciones independientes de 30 Hz en una banda de VHF de 112 a 117,9 Mhz emitida de una estación terrestre. Estas modulaciones son conocidas como fase de referencia  y fase variable, su diferencia de fase medida en grados sexagesimales, medida en el receptor a bordo del avión, corresponde a la marcación del mismo respecto del norte magnético.

La fase variable consiste en una modulación en amplitud de 30 Hz y la de referencia en una modulación en frecuencia del mismo rango, impresas sobre una subportadora de 9960 Hz modulada en amplitud. En los equipos más antiguos se genera la fase variable como una modulación espacial mientras que en los equipos más modernos un pseudo efecto Doppler genera la FM sobre la subportadora de la fase de referencia, mientras que la fase variable es una modulación en amplitud convencional


