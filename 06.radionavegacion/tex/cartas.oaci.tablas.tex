 \newcommand{\minitab}[2][1]{\begin{tabular}{#1}#2\end{tabular}}

% Cartas opcionales. Las seis cartas opcionales se producirán si, en opinión de las Autoridades aeronáuticas de los Estados, contribuyen a la seguridad, regularidad y eficacia de las operaciones de las aeronaves.

% Cartas condicionales. Las cinco cartas condicionales se producirán solamente si se cumplen determinadas condiciones o circunstancias.

\begin{landscape}

  \begin{table}%[!h]
    \centering
    \caption{Cartas aeron\'auticas OACI}
    \label{tab:cartas.aeronauticas.OACI}
    \begin{small}
      \begin{tabular}{|m{0.14\textwidth}|m{0.5\textwidth}|c|c|c|m{0.45\textwidth}|}
        \cline{3-5} 
	\multicolumn{2}{c|}{} & %\multirow{2}*[1.5cm]{
        \begin{turn}{90}
          \textbf{Obligatoria}
        \end{turn}
        % }
        &%\multirow{2}*[1.5cm]{
        \begin{turn}{90}
          \textbf{Opcional}
        \end{turn}
        % }
        &%\multirow{2}*[1.5cm]{
        \begin{turn}{90}
          \textbf{Condicional \,}
        \end{turn}
        % }
        & \multicolumn{1}{c}{}
        \\ \cline{1-2}\cline{6-6}
        \centering \textbf{Grupo} & \centering \textbf{Carta} & 
        & & &\multicolumn{1}{c|}{\textbf{Requerimientos}}\\ \hline
        \multirow{4}{0.12\textwidth}{ Planificaci\'on previa al vuelo} &
        \minitab[l]{1. Plano de obstáculos de aeródromo – Tipo A \\
          \quad (limitadores de utilización).} &\cellcolor{red} & & &
        Aeródromos con obstáculos destacados en las áreas de despegue y aterrizaje.
	\\
        &
        2. Plano de obstáculos de aeródromo – Tipo B. & &\cellcolor{blue} & & \\
        &
        3. Plano de obstáculos de aeródromo – Tipo C.& & &\cellcolor{green} & \\
        &
        4. Carta topográfica para aproximaciones de precisión. &\cellcolor{red} & & & Obligatoria en pistas con aproximaciones de precisión categorías II y III\\ \hline
        \multirow{4}{0.12\textwidth}{ En vuelo} &
        5. Carta de navegación en ruta. &\cellcolor{red} & & & 
	Zonas en las que se hayan establecido Regiones de Información de Vuelo (\ac{FIR})	\\
        &
        6. Carta de área. & & &\cellcolor{green} & \\
        &
        7. Carta de salida normalizada – Vuelo por instrumentos (SID). & & &\cellcolor{green} & \\
        &
        8. Carta de llegada normalizada – Vuelo por instrumentos (STAR). & & &\cellcolor{green} & \\
        &
        9. Carta de aproximación por instrumentos. &\cellcolor{red} & & &
        Aeródromos en los que se haya establecido tal tipo de aproximación \\
        &
        10. Carta de aproximación visual.  & & &\cellcolor{green} & \\ \hline
        \multirow{4}{0.12\textwidth}{Movimientos en tierra %de 
          %	las aeronaves en el aeródromo
        } &
        11. Plano de aeródromo/helipuerto &\cellcolor{red} & & & 
        En todos los que se utiliza regularmente la aviación civil internacional\\
        &
        12. Plano de aeródromo para movimientos en tierra.  & &\cellcolor{blue} & & \\
        &
        13. Plano de estacionamiento y atraque de aeronaves.  & &\cellcolor{blue} & & \\ \hline
        \multirow{4}{0.12\textwidth}{Navegación aérea visual, trazado de posiciones, planificación} &
        14. Carta aeronáutica mundial (escala 1:1.000.000). & \cellcolor{red}& & & En todas las zonas especificadas por la OACI\\
        &
        15. Carta aeronáutica (escala 1:500.000).  & &\cellcolor{blue} & & \\
        &
        16. Carta de navegación aeronáutica (escala pequeña). & &\cellcolor{blue} & & \\
        &
        17. Carta de posición.  & &\cellcolor{blue} & & \\ \hline
      \end{tabular}
    \end{small}
  \end{table}

\end{landscape}