% Radio Altimetro
% version 2018
%-------------------------------------------------------

El radioaltímetro fue inventado en 1924 por el ingeniero Lloyd Espenschied, si bien la compañía Bells Labs demor\'o 14 años en aplicar su diseño al uso en aeronaves.

El radioaltímetro de rango bajo (Low Range Radio Altimeter: LRRA) es un 
radar primario  vertical autocontenido que opera en la banda de 4,2 a 4,4 GHz.

El equipo aerotransportado comprende una antena transmisora/receptora, un transmisor/receptor y
una pantalla en cabina. La mayoría de los aviones están equipados con dos sistemas independientes. 
La energía del radar es dirigida, a través de una antena de transmisión, hacia el suelo;
parte de esta energía se refleja de nuevo desde la tierra y es recibida en la antena receptora
como se muestra en la Figura,


Se utilizan dos tipos de métodos LRRA para determinar la altitud de radio del avión.

\begin{itemize}
\item El m\'etodo de modulaci\'on de pulso mide  el tiempo transcurrido desde que la señal es transmitida 
hasta que es recibida, siendo este tiempo de retraso  directamente proporcional.
a la altura. 
\item El método de frecuencia modulada continua de onda (FM / CW) utiliza una se\~nal FM modificable
donde se fija la tasa de cambio. Una proporción de la señal transmitida se mezcla con
la señal recibida,  la señal de batido resultante es proporcional a la altitud.
\end{itemize}

La altitud medida por el instrumento se muestra en una instrumento dedicado o incorporado en una pantalla electrónica, como se observa en la Figura. 
Debe Tenerse en cuenta que la altitud de radio utilizada para aproximación y 
aterrizaje es s\'olo indicada a partir de los 750 m (2500 pies). 
La altura de decisión se selecciona durante los ajustes del ILS
