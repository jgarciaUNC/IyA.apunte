
\subsection{Radar}
\label{sec:radar}

El RADAR (Radio Detection And Ranking) es un sistema para detectar, mediante el empleo de ondas electromagnéticas, los obstáculos que se encuentran en las proximidades de un punto. Existen varios tipos y aplicaciones del radar, pero en la aeronáutica se utilizan principalmente para el control del espacio aéreo por parte de los centros de control y para la detección de obstáculos meteorológicos por parte de las aeronaves; también lo utilizan los centros de control meteorológico, pero la aeronáutica solo utiliza los resultados y las previsiones para la elaboración de los planes de vuelo. Dentro de la aeronáutica debemos separar el uso que del radar se hace en la aviación militar, que no es objeto de este estudio, y la utilidad que actualmente se le da en la parte civil de la aeronáutica.

CONTROL Y EXPLORACIÓN POR RADAR

Al estar el radar basado en la diferencia de tiempo entre la transmisión de una radioseñal y su reflexión por un objeto, esta reflexión tiene la capacidad de iluminar un punto en una pantalla, que es la que dispone el centro de control. La medición de esa diferencia de tiempo proporciona información sobre la distancia, el punto luminoso en la pantalla proporciona información al controlador. Los equipos de los centros de control son cada vez más precisos y capaces de identificar a las aeronaves individualmente, se emplean dos tipos de radares segun su precisión, radares de vigilancia para el control y radares de precisión para efectuar las ayudas en las aproximaciones instrumentales a las pistas de aterrizaje, consiguiendo que estas operaciones sean cada vez más rápidas y seguras.

Los centros de control disponen de grandes pantallas de radar donde la aeronave está representada por un punto brillante y al lado unos datos indicativos de altura, velocidad, compañia y n.o de vuelo, permitiendo al controlador una visión de conjunto de las aeronaves que sobrevuelan su zona de espacio y por lo tanto poder controlar sus posiciones relativas y sus maniobras. Los elementos con que cuentan los centros de control para explorar el espacio aéreo son unas grandes antenas, montadas sobre unas plataformas metálicas que les permite un giro de 360° y unas 15 revoluciones por minuto, desde las que lanzan la energia electromagnética, que se refleja en las aeronaves y núcleos torrnentosos, recibiéndola de nuevo la antena, calibrando asi la altura, azimut y distancia a la que se encuentra el objeto que refleja la señal emitida. RAD ETEQRQLQ’ QIQQ El principal objetivo del sistema es presentar datos o información de precipi- taciones de una forma fácil de interpretar, a fin de evitar la turbulencia asociada. Un objetivo secundario es proporcionar caracteristicas topográficas prominentes del te- rreno, tales como lagos, cerros o lineas costeras, etc., como ayuda a la navegación.


CONJUNTO DE ELEMENTOS DE UN SISTEMA DE RADAR

El sistema consta de los siguientes elementos: - Un Panel de Control. - Un transceptor. - Una antena. - Una guia de ondas. Desde el Panel de Control se selecciona basicamente el modo de operación, el ángulo de pitch de la antena, el rango de distancia de presentación, estabilización de antena y ganancia en la recepción para el modo de “ground MAP”. En los aviones equipados con EFIS, el radar es presentado en el Navigation Display y en el resto de aViones, el propio sistema de radar tiene una pantalla de Video exclusiva.

Un transceiver genera impulsos de radiofrecuencia y los transmite por una antena direccional. El tren continuo de pulsos transmitido direccionalmente se encontrará en su camino con particulas de agua o hielo de diferente masa, haciendo que parte de esa energia rebote y sea recibida por la antena y el receptor y tras su cronometración desde su transmisión y recepción, el radar calcula la distancia a la que se encuentra la masa detectada. Siendo proporcional la densidad de la masa detectada con el nivel de energia rebotada, el radar calcula la densidad de dicha masa, haciendo una presentación codificada en distintos colores, mostrando con ellos el nivel de probabilidad y magnitud de turbulencia, en el caso de ser atravesadas por el avión. Dicha presentación es la sección horizontal del área radiada por el radar. Los colores son desde un verde (el más suave) sin ninguna trascendencia, pasando por el amarillo que indica la posibilidad de turbulencias, el rojo, turbulencias de considerable magnitud y el magenta, siendo este el más alto nivel de turbulencias sobre la base de la densidad de la nube. El rango de alcance y presentación en pantalla es desde 2.5 a 250 NM (en algunos aviones llega el rango hasta 400 NM) con una capacidad selectiva de 15° a -15° de ángulo de Pitch, y 90° izquierdo hasta 90° derecho del eje longitudinal del avión.

La potencia de salida de la antena suele ser de 9345 MHz, con una cresta de 125 W. El ancho del impulso varia de ó useg. hasta 18 useg. Los impulsos de ó useg. se utilizan para retornar ecos entre 0 y 40 NM, mientras que los impulsos de 18 useg. se utilizan para rangos de más de 100 NM.

Las aeronaves que utilizan sistemas como el EFIS, ECAM, EICAS, etc., no tienen pantalla de radar especifica, presentan la información en las pantallas de instrumentos de navegación (ND) superpuesta sobre la poligonal de ruta en varios modos de navegación, Arco, Mapa, etc. En la figura anterior se presentan dos formas de información de radar en las pantallas de navegación con nubes y Windshear.


Gato Gutiérrez, F. (2013). Sistemas de aeronaves de turbina. Tomo I. San Vicente(Alicante), Spain: ECU. Recuperado de https://elibro.net/es/ereader/bmayorunc/62295?page=379.
