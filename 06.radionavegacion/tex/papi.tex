Seg\'un el OACI Anexo 14 5.3.5.23, las definiciones de los sistemas PAPI 
(	Precision Approach Path Indicator)
y APAPI 
(	Abbreviated Precision Approach Path Indicator )
son:

\begin{tabular}{m{0.1\textwidth}m{0.4\textwidth}m{0.4\textwidth}} \hline
	\rowcolor{yellow!30}
	& {\bf PAPI} & {\bf APAPI} \\
{\bf Descripci\'on} 
& El sistema PAPI consistirá en una barra de ala con
cuatro elementos de lámparas múltiples (o sencillas por pares)
de transición definida situados a intervalos iguales. El sistema
se colocará al lado izquierdo de la pista, a menos que sea
materialmente imposible.

{\it Nota.— Si la pista es utilizada por aeronaves que necesitan
guía visual de balanceo y no hay otros medios externos que
proporcionen esta guía, entonces puede proporcionarse una
segunda barra de ala en el lado opuesto de la pista.
}
  
& El sistema APAPI consistirá en una barra de ala
con dos elementos de lámparas múltiples (o sencillas por pares)
de transición definida. El sistema se colocará al lado izquierdo
de la pista, a menos que sea materialmente imposible.

{\it Nota.— Si la pista es utilizada por aeronaves que nece-
sitan guía visual de balanceo la cual no se proporciona
por otros medios externos, entonces puede proporcionarse
una segunda barra de ala en el lado opuesto de la pista.}

\\


\end{tabular}