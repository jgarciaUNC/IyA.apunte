\acrodef{2D}{Latitud y Longitud }
2D : Latitud y Longitud

\acrodef{3D}{Latitud, Longitud y Altitud}
3D : Latitud, Longitud y Altitud

\acrodef{FIR}{Flight Information Region }
FIR: Flight Information Region

\acrodef{IAC}{Instrumental Approach Charts}
IAC: Instrumental Approach Charts


% Glosario y abreviaturas:

% ACC: Area Control Center (Centro de control de área).
% AENA: Aeropuertos Españoles y Navegación aérea.
% AESA: Agencia Estatal de Seguridad Aérea.
% AIP: Aeronautical Information Publication. (Publicación de información aeronáutica).
% AMC: Airspace Management Cell (Célula de gestión del espacio aéreo).
% APP: APProach (Aproximaciòn).
% ATC: Air Traffic Control (Control de tránsito aéreo).
% ATFM: Air Traffic Flow Management (Organización de la afluencia de tránsito aéreo).
% ATIS: Automatic Terminal Information Service (Servicio automático de información termial).
% ATM: Air Traffic Management (Gestión del tránsito aéreo)
% ATZ: Aerodrome Traffic Zone (Zona de tránsito de aeródromo).
% AUP: Airspace Use Plan (Plan de utilización del espacio).
% AWY: AirWaY (Aerovía).
% CTA: ConTrol Area (Area de conrol).
% CTR: ConTRol zone (Zona de control).
% DME: Distance Measuring Equipment (Equipo Radiotelemétrico).
% FAA: Federal Aviation Administration.
% FAUP Forescast Airspace Use Plan (Previsión del plan de utilización del espacio aéreo).
% FIC: Flight Information Center (Centro de información de vuelo).
% FIR: Flight Information Region (Servicio de información de vuelo).
% FL: Flight Level. (Nivel de vuelo).
% FMP: Flow Management Position (Posición de gestión de afluencia).
% IIMC: Instrument Metereological Conditions. (Condiciones metereológicas de vuelo por istrumentos).
% IFR: Instrument Flight Rules (Reglas de vuelo por instrumentos).
% ILS: Instrument Landing System (Sistema de aterrizaje por instrumentos).
% OACI: Organización de Aviación Civil Internacional.
% RNAV Area navigation (Navegación de área).
% RVSM: Rerduced Vertical Separation Minimum (Separación vertical mínima reducida).
% SID: Standard Instrumet Departure (Salida standard por instrumentos).
% STAR: STandard instrument ARrival. (Llegada normalizada por instrumentos).
% TMA: TerMinal control Area (Area de control terminal).
% TWR: Tower (Torre de control).
% TWY: TaxiWaY. (Calle de rodae).
% UIR: Upper flight Information Region. (Región superior de información de vuelo).
% UTC: Universal Coordinated Time (Hora universal coordinada).
% VFR: Visual Flight Rules (Reglas de vuelo visual).
% VOR: VHF Omnidirectional Radio range (Radiofaro Omnidireccional VHF).