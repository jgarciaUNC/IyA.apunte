
transpondedor
Información del artículo
Categoría:
Redes de seguridad
Fuente del contenido:
Biblioteca SKY
Control de contenido:
EUROCONTROL
Descripción

Un transpondedor (XPDR) es un receptor/transmisor que generará una señal de respuesta tras una interrogación adecuada; el interrogatorio y la respuesta se realizan en diferentes frecuencias. (OACI)

Los transpondedores se utilizaron por primera vez para permitir a las autoridades militares identificar aviones amigos, que transmitían una señal codificada cuando eran interrogados por un radar militar. Esto se conocía como IFF (Identificación Amigo o Enemigo).

Radar de vigilancia secundario

Posteriormente, los transpondedores se han generalizado tanto en la aviación civil como en la militar. Ahora es una práctica estándar asignar un código de transpondedor específico a cada avión que vuela en espacio aéreo controlado para que el ATCO pueda identificar fácilmente un avión específico en una pantalla de radar abarrotada, utilizando SSR (radar de vigilancia secundario).

Por acuerdo internacional, el 2000 se utiliza para aeronaves a las que no se les ha asignado un código de transpondedor, aunque en algunas partes de Europa se utiliza el 7000 para este fin. Los detalles de los códigos estándar en diferentes países se pueden encontrar en  las Publicaciones de Información Aeronáutica (AIP) nacionales .

En caso de emergencia se utilizan códigos especiales, como sigue:

    secuestro (7500);

    pérdida de comunicación  (7600); y,

    emergencia general (7700).

Actualmente existen sistemas como el Equipo de detección de superficie aeroportuaria – Modelo X ( Equipo de detección de superficie aeroportuaria, Modelo X (ASDE-X) ) y el Sistema avanzado de control y guía del movimiento en la superficie ( Sistema avanzado de control y guía del movimiento en la superficie ) que utilizan retornos de transpondedor de ambos aeronaves y vehículos de servicio aeroportuario con transpondedores instalados para mejorar la seguridad y eficiencia del control del movimiento en superficie. Varios aeropuertos grandes han incluido información en las transmisiones del  Servicio automático de información de terminales (ATIS)  cuando se requiere que el transpondedor esté activo para las operaciones de taxi. Además, en determinados aeropuertos se utiliza una combinación optimizada localmente de tecnologías disponibles, es decir, multilateración aeroportuaria, radares de movimiento en superficie y Transmisión automática de vigilancia dependiente (ADS-B) , habilita sistemas A-SMGCS y operaciones aeroportuarias integradas. Esto podría incluir la disponibilidad de una visualización adecuada de la información de vigilancia en una pantalla consolidada en forma de mapa en movimiento en las cabinas de vuelo y en los vehículos de superficie.
Modo A, C, S e Ident

Las aeronaves civiles podrán estar equipadas con transpondedores capaces de funcionar en diferentes modos:

    El equipo en Modo A transmite únicamente un código de identificación.
    El equipo en Modo C  permite al ATCO ver la altitud de la aeronave o el nivel de vuelo automáticamente.
    El equipo Modo S  tiene capacidad de altitud y también permite el intercambio de datos.

El equipo en Modo C o S es un requisito obligatorio para muchas áreas concurridas del espacio aéreo controlado.

Los transpondedores tienen una función de "identificación" que hace que la respuesta del radar de la aeronave se destaque cuando el piloto opera el interruptor de identificación en la cabina. Esto sólo debe operarse a pedido del ATC.
Requerimientos legales

El Reglamento (UE) n.º 1207/2011 exige que todos los vuelos que operen como tráfico aéreo general de conformidad con las normas de vuelo por instrumentos dentro de la UE estén equipados con transpondedores en modo S.
Uso del transpondedor en cajeros automáticos

Los transpondedores se utilizan en cajeros automáticos para diversos fines, siendo los más destacados:

    Identificación de aeronaves;
    Mejorar la conciencia situacional de los controladores;
    Desarrollo de herramientas ATC y redes de seguridad (por ejemplo, AMAN, MTCD, STCA, etc.).

Transpondedores y ACAS

El funcionamiento del sistema para evitar colisiones aerotransportadas (ACAS)  requiere que ambas aeronaves (el interrogador y el objetivo) estén equipadas con transpondedores operativos. Una aeronave equipada con ACAS recibirá la siguiente información dependiendo del tipo de transpondedor con el que esté equipada la aeronave objetivo:
Avión objetivo equipado con: 	Aviones interrogadores equipados con ACAS:
Sólo transpondedor modo 'A' 	No rastreará el objetivo
Transpondedor en modo 'A/C' sin informes de altitud 	Por debajo de FL 155: recibe únicamente aviso de tráfico (TA) (no se mostrará ninguna flecha de altitud ni de tendencia)

Por encima de FL 155: no mostrará el objetivo
Transpondedor modo 'C' o 'S' 	Recibe asistencia técnica y aviso de resolución vertical (RA)
ACAS 	Recibe TA y RA vertical coordinada