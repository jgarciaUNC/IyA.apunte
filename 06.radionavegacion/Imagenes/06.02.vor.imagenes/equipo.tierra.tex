\section*{Equipo de tierra}

\subsection*{Principios de funcionamiento }

El transmisor es de tipo AM de diseño estándar excepto en el hecho de que el modulador debe ser capaz de trabajar a  frecuencias por encima de los 10 kHz. Para las ayudas en ruta, la potencia de salida es de 200 W, para servicio en aeródromos sólo se necesita 50 W.

La operación de un equipo VOR de tierra está basada en Ia diferencia de fase entre dos señales que emite: una de referencia y otra variable. Cada grado de variación de fase entre las señales, representa un grado de variación de posición del avión.

Los radiales de un VOR son infinitos, pero el equipo de a bordo solo es capaz de diferenciar 360 de ellos.

En una estación VOR, un sistema de monitores y dos transmisores, aseguran un servicio continuo de funcionamiento. Si la señal del equipo se interrumpe por cualquier causa, o varían sus fases, el sistema de monitores desconecta el equipo defectuoso, conectando a su vez un transmisor auxiliar y excitando una alarma en el panel de control que indica un fallo en el sistema. En la Fig. 3-1 1 , puede verse una estación de tierra. 

El equipo transmisor trabaja en VHF en la banda de 112 Mhz a 118 Mhz, en frecuencias que terminan en décimas pares o impares, y centésimas impares. Se podrán usar frecuencias comprendidas entre 108 Mhz y 112 Mhz cuando:

\begin{itemize}
\item Se usen en VOR de cobertura limitada únicamente

\item Se usen solo frecuencias que terminen bien en décimas pares o centésimas impares de Mhz

\item No se utilicen estas frecuencias para el sistema ILS.

\item No se ocasionen interferencias al ILS
\end{itemize}


\subsection*{Cono de silencio}

En la emisión de las estaciones VOR se producen ciertas zonas ciegas donde la señal es nula. A estas zonas se las Ilama conos de silencio, y se encuentran localizadas sobre la estación. Cuando la aeronave la esté sobrevolando, no recibirá ningún tipo de señal. La amplitud de la zona de silencio, debido a su forma de cono invertido, se incrementa con la altura. De esta manera, un avión volando a 20.000' sobre una instalación VOR, permanecerá más tiempo en el cono de silencio que otro avión que lo esté haciendo a 10.000'.

Clasificación y tipos de estaciones VOR

La clasificación de las estaciones VOR se efectúa de acuerdo con la altitud y distancia libre de interferencias a la que éstas pueden recibir­se. Existen dos criterios sobre el particular: el americano y el de OACI.

La clasificación americana de la F.A.A. es la siguiente:
\begin{itemize}
\item \textbf{T-VOR. VOR terminal o de recalada:} Las condiciones operativas de este primer tipo de VOR son tales que no debe ser usado para la navegación si la aeronave que desea sintonizar­lo, está a más de 25 NM de Ia estación y a una altitud superior a 12.000'. A partir de esta distancia y altitud, sus indicaciones no son de fiar.
Los VOR de recalada se usan principalmente como ayuda a la aproximación a los aeropuertos, y nunca como ayudas de ruta.


\item \textbf{L-VOR. VOR de baja altitud:}  Este tipo de estación puede usarse con seguridad hasta una distancia de 40 millas náuticas y una altitud de 18.000 pies. Puede usarse, además de come ayuda a la aproximación como apoyo en ruta.


\item \textbf{H-VOR. VOR de gran altitud:} El H-VOR tiene un alcance de unas 40 millas náuticas por debajo de 18.000 pies y de 130 millas náuticas por encima de esta altitud, con un máximo de 156 millas náuticas a  75000 pies. Los alcances de los distintos tipos de VOR no deben confundirse con una mayor o menor potencia de emisión de las estaciones de tierra, pues ésta es prácticamente la misma para todos, situándose alrededor de los 200 W. 

\end{itemize}

Según OACI, únicamente hay dos tipos de instalación VOR. 

\begin{itemize}
\item \textbf{VOR-A:} Una aeronave recibirá las señales de este tipo de VOR, hasta una distancia de 100 millas náuticas por lo menos, y hasta un ángulo de elevación de 40 grados, siempre que no existan obstáculos entre la estación y dicha aeronave.

\item \textbf{VOR-B:} Esta estación VOR será recibida a una distancia de 25 millas náuticas y con un ángulo de 40 grados por lo menos.

\end{itemize}

Actualmente, existe gran cantidad de instalaciones VOR, por lo que en determinados Iugares, a lo Iargo de una ruta, podría darse el caso de que dos estaciones, emitiendo en Ia misma frecuencia o en frecuencias muy cercanas, se interfirieran. 

En vistas a que esto no suceda, Ias áreas en las que estas interferencias son posibles, vienen indicadas en las cartas de navegación con eI símbo­lo MAA seguido de unas cifras que indican una altitud. La MAA o Altitud máxima autorizada, asegura la nítida recepción de una señal VOR  sin interferencias, y por supuesto, guardando la mínima separación de seguridad con el terreno.

La recepción de una señal interferida se hará evidente por falsas indicaciones en el receptor VOR, por oscilaciones de los indicadores y por silbidos agudos.

La única corrección posible a este inconveniente, es la sintonización e otra estación VOR que convenga a la ruta que se está volando. Realmente es muy difícil que dos equipos VOR cercanos transmitan en la misma frecuencia, pero en zonas de gran densidad de instalaciones, puede Llegar a suceder.

