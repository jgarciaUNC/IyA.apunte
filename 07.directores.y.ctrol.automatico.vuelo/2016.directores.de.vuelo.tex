 \documentclass[a4paper,12pt,twoside]{article}

\usepackage{../IyA.estilo.2015}
\usepackage{acronym}
\usepackage{caption}

%-------------------para incluir archivos graficos .gif------%
\epstopdfDeclareGraphicsRule{.gif}{png}{.png}{%
  convert gif:#1 png:\OutputFile
}
\AppendGraphicsExtensions{.gif}



%------------------Define box and box title style
\tikzstyle{mybox} = [draw=red, fill=yellow!20, very thick,
    rectangle, rounded corners, inner sep=10pt, inner ysep=20pt]
\tikzstyle{fancytitle} =[fill=blue!80, text=white]

\tikzstyle{grupos} = [draw=blue, fill=green!20, very thick,
    rectangle, rounded corners, inner sep=10pt, inner ysep=20pt]
\tikzstyle{fancytitle.grupos} =[fill=blue, text=white, ellipse]


\newcommand{\fullref}[1]{\ref{#1} de la p\'agina \pageref{#1}}

%-------------------------------------Encabezado
\fancyhead[RO,LE]{\small{{\bf \large Instrumentos y Avi\'onica} \\
\textcolor{blue}{\bf Directores de vuelo}\\
P\'agina \thepage \ de \pageref{LastPage}
                 }
                 }

\fancyhead[RE,LO]{
	\includegraphics[height=1.5cm]{../imagenes/UNC-400-anios.png}
	\includegraphics[height=1.2cm]{../imagenes/logo.jpg}
	\includegraphics[height=1.2cm]{../imagenes/fcefyn.png}
	\includegraphics[height=1.2cm]{../imagenes/dpto-aero-logo.jpg}
}

%----------------------------------Caratula
\title{Directores de vuelo}
\author{Ing. Jorge Garcia (jgarcia@efn.uncor.edu)}
\date{\today}

\begin{document}

\renewcommand{\tablename}{Tabla}

\newcommand{\ESPACIO}{\rule{0in}{3ex}}

\thispagestyle{fancy}
\maketitle

\thispagestyle{fancy}
\tableofcontents


\section{Finalidad y operaci\'on}
\label{sec:finalidad.y.operacion}


Estos sistemas tienen como finalidad dirijir al piloto para que este efect\'ue en forma correcta las maniobras de vuelo seg\'un el modo de operaci\'on elejido, adem\'as cumple las funciones b\'asicas de indicaci\'on de actitud y rumbo.

Existen diferentes tipos de directores de vuelo en cuanto a la forma de 
indicaci\'on y selecci\'on de modos pero cumplen la misma funci\'on.

\section{Componentes de cabina}
\label{sec:componentes.cabina}

\subsection{Indicador y Director de Actitud }
\label{sec:adi}

El ADI (Attitude Director Indicator)
 posee una esfera indicadora de actitud y punteros de 
mando (lateral y vertical), que dan al piloto la informaci\'on requerida
para interceptor y mantener una determinada senda de vuelo que, junto
con sus dem\'as componentes, son:

\begin{figure}[!h]\centering
  \includegraphics[width=0.98\textwidth]{imagenes/adi.png}
  \caption{Imagen de ADI (gentileza Sperry)}
\label{fig:adi.sperry}
\end{figure}

\begin{itemize}
	\item {\bf Esfera de actitud: } parte m\'ovil donde est\'a simbolizado
	el horizonte terrestre, sus movimientos con respecto a un avi\'on simb\'olico
	fijo, indican la actitud en alabeo y cabeceo de la aeronave.

        \item {\bf Avi\'on simb\'olico en miniatura: } s\'imbolo fijo al centro del
	instrumento y con respecto a su centro dan su indicaci\'on las barras de mando, 
	adem\'as de cumplir la funci\'on indicada en el punto anterior.

        \item {\bf Indice de actitud en alabeo: } se mueve con la esfera indicando
	exactamente las gradas en esta actitud.

        \item {\bf Barras de mando: } su funci\'on es dirigir al piloto para interceptar
	y mantener una predeterminada senda de vuelo con la condici\'on de ``volar''
	el peque\~no avi\'on simb\'olico hacia las barras de comando y tratar de
	mantener el centro de este en la intersecci\'on de las barras.
	La barra de mando vertical comanda las actitudes a tomar en alabeo y la
	horizontal en cabeceo.
	Dado que el ``avi\'on'' es fijo, al tomar la actitud correcta las barras que
	son m\'oviles se posicionar\'an en el centro de este s\'imbolo.

        \item {\bf Escala de senda de planeo: }
	el \'indice muestra la desviaci\'on del avi\'on del centro del haz de senda de 
	planeo (Glide Slope) cuando es sintonizada la frecuencia del ILS y la se\~nal
	recibida es v\'alida. 
	Si el avi\'on est\'a volando por debajo del haz, el \'indice se ubicar\'a
	en la parte de arriba de la escala.
	Una indicaci\'on al punto del \'indice representa, aproximadamente,
	 una desviaci\'on de 0,4º
	respecto de la l\'inea de control del haz.

        \item {\bf Barra de radio altura: }
	con el sistema de radioalt\'imetro en funcionamiento, esta barra aparece
	a la vista a los 61 m de altura respecto al terreno, movi\'endose hacia el
	avi\'on en miniatura seg\'un se desciende 

        \item {\bf Localizador expandido: }
	provee de una muy sensible indicaci\'on de la posici\'on del avi\'on
	respecto a la linea central del localizador, siendo utilizado en la
	aproximaci\'on final, dada su sensibilidad.

        \item {\bf Inclin\'ometro: }
	suministra al piloto una indicaci\'on convencional de los deslizamientos
	y derrapes del avi\'on. Al mantener la esfera indicadora centrada,
	se aseguran maniobras coordinadas.

        \item {\bf Perilla de ajuste de actitud de cabeceo: }
	permite posicionar la barra horizontal para que esta comande una
	predeterminada actitud de cabeceo durante una picada o trepada.

        \item {\bf Pulsador de prueba de actitud: }
	opera como una autoprueba del indicador de actitud. Cuando es pulsada
	la esfera se posiciona arrastrando un alabeo de 20º a la derecha
	y un cabeceo de 10º en trepada, apareciendo en este caso una bandera
	de advertencia de actitud erronea.

        \item {\bf Llave de erecci\'on r\'apida del gir\'oscopo: }
	no se encuentra sobre el instrumento, ubic\'andose en el tablero.
	Cuando esta, que es cargada a resorte, es pulsada y mantenida, 
	permite la erecci\'on del gir\'oscopo a una velocidad, aproximada,
	de 2º/minuto, cuando este gire a su m\'axima velocidad.
	Esta llave debe ser accionada cuando el avi\'on se encuentra
	nivelado, y se utiliza en caso de que el gir\'oscopo haya salido
	de su plano de referencia, como puede ocurrir por haberse tomado
	actitudes que superan las posibilidades del sistema.

\end{itemize}

\subsection{Indicador de Situaci\'on Horizontal}
\label{sec:indicador.situacion.horizontal}

El HSI (Horizontal Situation Indicator)
 provee adem\'as de la indicaci\'on de rumbo del avi\'on
una indicaci\'on pictogr\'afica que representa la posici\'on
del avi\'on respecto a un localizador VOR, y una indicaci\'on de la
posici\'on del avi\'on respecto a la senda de planeo.

\begin{figure}[!h]
  \centering
  \includegraphics[width=0.98\textwidth]{imagenes/hsi.png}  
  \caption{Imagen HSI (gentileza Sperry)}
  \label{fig:hsi.sperry}
\end{figure}


La descripci\'on de cada uno de sus componentes:

\begin{itemize}
	\item {\bf S\'imbolo del avi\'on:} se encuentra fijo e indica la
		posici\'on de la aeronave respecto a un curso de radio
		y a un cuadrante m\'ovil indicador del rumbo. Esta fijado
		al vidrio del instrumento.

        \item {\bf Cuadrante rotante de rumbo:} provee ula informaci\'on de un
	comp\'as magn\'etico girosc\'opico, girando seg\'un los rumbos tomados por
	la aeronave a trav\'es de los 360º.

        \item {\bf Indice principal de rumbo: } 
	es fijo y marca sobre el cuadrante el rumbo del avi\'on.

        \item {\bf Marcas de azimut: }
	se encuentran fijas a una diferencia de 45º a trav\'es de los 360º

        \item {\bf Indice y perilla de rumbo selectado: }
	por medio de la perilla se posiciona el \'indice en la carta al rumbo
	deseado. La diferencia angular entre el rumbo del avi\'on y el preselectado
	

        \item {\bf Puntero y perilla de curso: } este se osiciona en el cuadrante de 
	rumbo por medio de la perilla de curso de manera que
	coincida con el radial de VOR o curso de localizador deseado.
	Igual que el \'incide de rumbo selectado, el puntero de curso es posicionado
	sin afectar la indicaci\'on del cuadrante de rumbo, pero al girar este
	lo hace de igual forma.
	El puntero provee en forma continua la informaci\'on del error de curso
	al piloto y a la computadora del sistema, de manera que 
	cuando es selectado un modo de radio, la barra vertical en el ADI
	dirije al piloto para que controle los comandos y asuma las actitudes
	de alabeo que lo llevar\'an a interceptar y mantener el curso de radio
	selectado, todo esto con la condici\'on de matener la barra vertical en el 
	ADI centralizada.

      \item {\bf Barra de desviaci\'on de curso: }
	representa la l\'inea central del curso selectado de VOR o localizador, 
	el s\'imbolo del avi\'on muestra la posici\'on relativa del mismo
	respecto al curso selectado. Esta barra se mueve paralelamente al puntero
	de curso seg\'un la se\~nal de radio recibida.

      \item {\bf Puntos de desviaci\'on de curso: }
	en la operaci\'on de VOR, cada punto representa 5º de desviaci\'on de la l\'inea
	central y en ILS, es 1º.

      \item {\bf Indicador de ``\emph{hacia-de}'': }
	son dos banderas que aparecen, una por vez e indicando si se est\'a alejando
	de la estaci\'on ``\emph{de}'' o si se va hacia ella ``\emph{hacia}''.

      \item {\bf Disco m\'ovil de curso: }
	es el sost\'en f\'isico del puntero de curso, barra de desviaci\'on de curso,
	indicador de ``\emph{hacia-de}'', y puntos de desviaci\'on de curso.
	Este disco gira por medio de la perilla de curso, arrastrando los elementos
	sostenidos en \'el. Est\'a pintado de manera que se confunde con el
	cuadrante de rumbo, dado que no es un elemento indicador.

      \item {\bf \'Indice y puntos de desviaci\'on de senda de planeo: }
	repite la informaci\'on de desviaci\'on de senda de planeo dada por el ADI.
	El \'indice se muestra al sintonizar una frecuencia de localizador.
	Cuando el avi\'on se encuentra por debajo de la senda de planeo, 
	el \'indice se encuentra en la parte de arriba de la escala.
	Cada punto representa $0.4$º de desplazamiento.

      \item {\bf Anunciador de sincronismo:}
	es una marca de punto (.) o cruz (x) que aparece en una peque\~na ventanilla
	indicando el sentido del error del rumbo indicado por el cuadrante de
	rumbo respecto al verdadero rumbo magn\'etico del avi\'on.
	Cuando el rumbo indicado es el verdadero las marcas de punto y cruz
	aparecer\'an alternativamente en la ventanilla indicando el sincronismo
	entre el cuadrante de rumbo y el sistema girosc\'opico autocorregido.

      \item {\bf Llave de ``\emph{esclavo - libre}'': }
	no se encuentra en el HSI, siendo ubicada en un lugar conveniente en el tablero
	y se selecta por medio de ella el modo de trabajo del comp\'as 
	girosc\'opico (libre o esclavizado seg\'un el rumbo magn\'etico).

      \item {\bf Llave de incremento (INC) - decremento (DEC): }
	no se encuentra 
	en el HSI, ubic\'andose cerca de la llave de   ``\emph{esclavo - libre}''.
	Con esta se posiciona el cuadrante de rumbo para obtener la
	sincronizaci\'on del sistema en el modo ``\emph{esclavo}'' y para
	modificar el rumbo indicado en el modo ``\emph{libre}''.

\end{itemize}

\subsection{Control - Computador}
\label{sec:control.computador}

Este elemento combina los datos de rumbo, actitud, altitud y receptor de 
navegaci\'on en se\~nales computadas que comandar\'an las barras
directoras del ADI.
Contiene en su frente los botones por medio de los cuales se
seleccionan el o los modos de operaci\'on deseados.

Los modos que se encuentran en operaci\'on son enunciados al iluminarse
el bot\'on correspondiente.


\begin{figure}[!h]
  \centering
  \includegraphics[width=0.6\textwidth]{imagenes/computadora_sperry.png}  
  \caption{Imagen Control-computador (gentileza Sperry)}
  \label{fig:computadora.sperry}
\end{figure}

\section{Modos de operaci\'on}
\label{sec:modos.operacion}

Este sistema utiliza los datos de VOR-Localizadores y de Senda de Planeo,
proporcionados por los receptores de navegaci\'on corrientes.
Utiliza tambi\'en datos de altitud de un sensor de altura barom\'etrico
propio (adem\'as del de sistema de radio-alt\'imetro), datos de rumbo
desde un gir\'oscopo direccional y de actitud desde un gir\'oscopo
vertical, elementos estos que pertenecen al mismo sistema.

Todas estas informaciones son computadas siendo finalmente enviado su
resultado a comandar las barras directoras del sistema para guiar al
piloto en las maniobras que debe efectuar, para mantener o tomar la
senda de vuelo selectada, tanto para la navegaci\'on como para la
aproximaci\'on de aterrizaje.

Los modos de operaci\'on que puede selectar el piloto son los siguientes:

  \begin{tabular}{lm{3mm}llm{3mm}l}
\rowcolor{cyan!10}
    	SBY &  & Preparado &
	GO AROUND &  & Escape \\
\rowcolor{yellow!10}
	ALT &  & Mantenimiento de altura    &
	PAT &  & Ajuste de actitud en cabeceo \\
\rowcolor{cyan!10}
	HDB &  & Rumbo selectado &
	V/L &  & VOR - Localizador \\
\rowcolor{yellow!10}
        GS ARM & & Armado Pendiente de Planeo &
	GS & & Pendiente de Planeo \\
\rowcolor{cyan!10}
	GS EXT & &  &
	REV & & Curso opuesto \\
  \end{tabular}

\subsection{Modo Preparado (SBY)}
\label{sec:modo.sby}
Mediante el mismo el sistema es puesto en condiciones de operar cuando sea
requerido y encontr\'andose las barras del ADI fuera de la vista,
operando \'este como un indicador de referencia de actitud.
Este sistema est\'a energizado si lo est\'a el sistema de corriente
alterna del avi\'on, por lo tanto permite la selecci\'on de cualquier modo siempre que las banderas de precauci\'on respectivas se encuentren fuera de la
vista.

Se selecciona el moto SBY presionando la tecla correspondiente
en tablero, al seleccionarla se iluminan los dem\'as modos
como prueba de l\'amparas y al soltarlo solo SBY queda iluminado.

Al seleccionarse otro modo las barras directoras responden a las salidas
de la computadora, luego el piloto debe ``volar'' el avi\'on en miniatura
del ADI hacia las barras directoras e interceptarlas. Al mantener esta
condici\'on ser\'an efectuadas las maniobras necesarias para interceptar
y/o mantener un curso deseado.

\subsection{Modo Ajuste Actitud de Cabeceo (Pitch Attitude Trim, PAT)}
\label{sec:modo.ajuste.pitch}

Permite selectar un \'angulo de trepada o descenso por medio de la perilla
de ajuste de ajuste de actitud de cabeceo, que se encuentra en el frente y
en el extremo inferior izquierdo
del ADI (Figura \ref{fig:adi.sperry}, Pitch attitude trim knob).

Al presionar PAT, debe mantenerse encendida la luz de dicho modo
para que se confirme el mismo, la barra de comando de cabeceo del
ADI se ubica de manera que el piloto tome la actitud de cabeceo
deseada.

El modo PAT puede ser utilizado con cualquier otro modo de 
control de alabeo (p.e. el HDG) pero
resulta incompatible con otros modos de cabeceo (p.e. GO AROUND, ALT).

Si la se\~nal de cabeceo se torna inv\'alida, la barra de comando saldr\'a 
de vista y el bot\'on del modo permanecer\'a iluminado.

\subsection{Modo Rumbo Selectado (Heading HDG)}
\label{sec:hdg}

Se utiliza para interceptar y mantener un rumbo de vuelo deseado.
Al presionar HDG se debe iluminar el bot\'on, confirmando la
operaci\'on del mismo.
Si el modo resultara no v\'alido, la barra de comando de alabeo
saldr\'a de vista pero el bot\'on continuar\'a iluminado.

Utilizando la perilla de rumbo selectado ubicada en  el
HSI (Figura \ref{fig:hsi.sperry}, Heading knob),
se posiciona el \'indice de rumbo selectado en el rumbo
deseado 
(Figura \ref{fig:hsi.sperry}, Heading bug). 
Este debe ser menor a 170º respecto al rumbo actual de la aeronave.

La computadora controlar\'a los movimientos de la barra de
alabeo en el ADI a fin de dirigir al piloto para corregir
el alabeo del avi\'on de forma de interseptar el curso elegido
sin sobrepasamiento.

Para prevenir actitudes extremas, la computadora limita los \'angulos
de alabeo a un m\'aximo de 30º.

Si el modo HDG es selectado desde el modo SBY, en el ADI s\'olo
aparecer\'a la barra de alabeo. 
Si el modo es selectado con el modo PAT o ALT, aparecer\'a adem\'as
la barra de cabeceo.

\subsection{Modo mantenimiento de altura (ALT)}
\label{sec:alt}

Brinda la posibilidad de mantener una altura barom\'etrica deseada,
la cual ser\'a la presente al momento de selectar este modo.

Entonces, se nivela la aeronave a la altura deseada y se oprime
el modo ALT, confirm\'andose su operaci\'on por permanecer el 
bot\'on iluminado.

La computadora brindar\'a la informaci\'on necesaria para
posicionar la barra de cabeceo en el ADI, de forma de 
que el piloto tome las actitudes necesarias para mantener
la altitud barom\'etrica deseada.

Para prevenir acciones extremas, el comando del \'angulo de
cabeceo est\'a limitado a $\mp$ 10º.

En caso de alg\'un mal funcionamiento, la barra directora
de cabeceo en el ADI saldr\'a de vista y la luz de ALT
se apagar\'a en el teclado.

Adem\'as, a los efectos de proteger el elemento sensor de altura,
el modo se cancelar\'a al producirse un cambio sustancial respecto
a la altura selectada, desapareciendo la barra de control
de cabeceo.
Por ejemplo, al desviarse a nivel del mar en $\mp$ 400 pies (120 m)
o en $\mp$ 1200 pies (360 m) a 40000 pies (12000 m) de altura. 


\subsection{Modo Escape (GO AROUND)}
\label{sec:go.araund}

Este modo cancela todos los otros modos y provee un comando de trepada
prefijado, juntamente con un comando de nivelaci\'on de las alas.

El modo GO AROUND se selecta presionando el bot\'on correspondiente
sobre el teclado de la computadora o mediante un bot\'on en el
volante de mando.

En ocasi\'on de una aproximaci\'on fallida, se selecta este modo
indicando las barras del ADI la actitud a tomar.

Si el gir\'oscopo vertical asociado al sistema entrega datos
incorrectos o no v\'alidos, las barras en el ADI salen de
la vista mientras que el bot\'on GO AROUND permanecer\'a iluminado.


\subsection{Modo VOR - Localizador (V/L)} \\
\label{sec:V_L}

Permite mediante la barra de alabeo del ADI, interceptar y mantener
un rumbo deseado, operaci\'on que se efectuar\'a suavemente
limitando la computadora los \'angulos de alabeo a $\mp$ 30º.

\subsection{Armado Pendiente de Planeo}
\label{sec:GS.arm}

Por medio de este modo se prepara al sistema para la captura
autom\'atica del haz de pendiente de planeo 
cuando el avi\'on se aproxime al centro del haz de pendiente
de planeo desde abajo de \'este, si el avi\'on se encuentra
por arriba la luz del bot\'on GS-ARM se apagar\'a y la
computadora activar\'a autom\'aticamente el modo GS.

\subsection{Pendiente de Planeo}
\label{sec:gs}

Permite al piloto efectuar la captura de la pendiente de planeo
desde arriba o abajo del haz. La barra de mando de cabeceo mostrar\'a
la actitud a tomar, se encuentre el avi\'on arriba o abajo de la pendiente
de planeo.
La computadora limita los \'angulos de cabeceo a $\mp$ 10º aproximadamente.
La barra de mando de alabeo permitir\'a mantener la l\'inea central del
localizador.

\subsection{Modo anunciador ampliaci\'on de indicaci\'on en pendiente de planeo (GS EXT)}
\label{sec:gs.ext}

Provee un ajuste autom\'atico de ganancia para compensar el estrechamiento
del haz de pendiente de planeo, \'este modo es enganchado
autom\'aticamente al detectarse el marcador medio o a los 250 pies (76,2 m)
de altura.

\subsection{Modo Curso Opuesto (REV)}
\label{sec:rev}

Este modo da la posibilidad de volar el curso de localizador opuesto.

\section{Incorporaci\'on del sistema de radioalt\'imetro}
\label{sec:incorporacion.sistema.radio.altimetro}

El radioalt\'imetro provee una indicaci\'on de altura absoluta en un indicador,
pero adem\'as se interconecta con el director de vuelo a los fines
de manejar la barra de radioaltura en el ADI.

Cuando el sistema de radioalt\'imetro funciona, la barra de radioaltura aparece a la vista
en la parte inferior del instrumento a los 61 (sesenta y un) metros y, a medida que
el avi\'on se acerca a tierra, esta barra sube simulando el acercamiento de la aeronave
a tierra mediante el acercamiento relativo del avi\'on miniatura del ADI
a la barra en miniatura que simula el terreno.

Al tocar tierra la aeronave, la barra de altura del instrumento toca la parte
inferior del avi\'on simb\'olico del ADI.

El sistema de radioaltimetro est\'a compuesto por dos (2) antenas, una transmisora
y otra receptora, un transceptor y un indicador.
Su principio de funcionamiento se basa en la emisi\'on de ondas de UHF, moduladas
en frecuencia, por la antena transmisora, las cuales son reflejadas por la superficie
terrestre y receptadas por la segunda antena. La diferencia de frecuencia entre
las ondas emitidas y las reflejadas es representativa de la altura a la cual se
encuentra la aeronave del suelo, y es indicada en el instrumento.

En otros sistemas se utiliza la modulaci\'on de pulsos.



\end{document}
