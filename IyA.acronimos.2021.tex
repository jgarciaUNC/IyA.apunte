%****** Acronyms and abbreviations in avionics ******
%***** [edit] A *****

\DeclareAcronym{2D}{
	short = 2D,
	long  = Latitud y Longitud }

%2D : Latitud y Longitud

\DeclareAcronym{3D}{
	short = 3D,
	long  = Latitud\, Longitud y Altitud}
%3D : Latitud, Longitud y Altitud



% \acrodef{AGPL}{Affero General Public License}
% AGPL: Affero General Public License.
\DeclareAcronym{A/A}{
	short = A/A,
	long  = Air to Air
}

\DeclareAcronym{A/S}{
	short = A/S,
	long  = Air to Surface
}



%     * ACARS: Aircraft Communications Addressing and Reporting System
\DeclareAcronym{ACARS}{
  short = ACARS ,
  long  = Aircraft Communications Addressing and Reporting System
}

%     * ACAS: Airborne Collision Avoidance System
%     * ACP: Audio control panel
%     * ACS: Audio control system

%     * ADC: Air data computer
\DeclareAcronym{ADC}{
        short = ADC,
        long  = Air Data Computer
}

%     * ABAS: Aircraft-based augmentation system
\DeclareAcronym{ACFT}{
	short = ACFT,
	long  = Aircraft
}

%     * ADF: Automatic direction finder
\DeclareAcronym{ADF}{
	short = ADF,
	long  = {Buscador Autom\'atico de Direcci\'on, Automatic Direction Finder} 
}

%     * ADI: Attitude director indicator
\DeclareAcronym{ADI}{
	short = ADI,
	long  = Attitude Director Indicator
}

%     * ADIRS: Air Data Inertial Reference System
%     * ADIRU: Air data inertial reference unit
%     * ADM: Air data module
%     * ADS: Either; Automatic Dependent Surveillance or air data system
%     * ADS-A: Automatic Dependent Surveillance-Address
%     * ADS-B: Automatic Dependent Surveillance-Broadcast
\DeclareAcronym{ADPCM}{
	short = ADPCM,
	long  = {Modulación por codificación de impulsos diferencial adaptativa, Adaptive Differential     Pulse Code Modulation}
}

%     * AFCS: Automatic flight control system
\DeclareAcronym{AFCS}{
	short = AFCS,
	long  = Automatic Flight Control System
}
%     * AFD: Autopilot flight director
%     * AFDC: Autopilot flight director computer
%     * AFDS: Autopilot flight director system
%     * AFIS: Either; Automatic flight information service or airborne flight
%       information system
%     * AGACS: Automatic ground–air communications system, is also known as ATCSS
%       or data link
%     * AGDL: Air-Ground Data Link
%     * AGC: Automatic gain control
\DeclareAcronym{AGL}{
	short = AGL,
	long = Above Ground Level}

%     * AHC: Attitude heading control
%     * AHRS: Attitude and Heading Reference Systems
\DeclareAcronym{AHRS}{
	short = AHRS,
	long  = Attitude and Heading Reference Systems }

%     * ADAHRS: Air data and attitude heading reference system
\DeclareAcronym{ADAHRS}{
	short = ADAHRS,
	long  = Air Data and Attitude Heading Reference System}


\DeclareAcronym{AI}{
	short = AI,
	long  = Attitude Indicator}

\DeclareAcronym{AIDS}{
short = AIDS, 
long = Aircraft Integrated Data System}
%       AIDS: Aircraft Integrated Data System

%     * ALC: Automatic level control
%     * ALT: Either; altimeter or altitude
%     * ALT Hold: altitude hold mode
%     * ALTS: Altitude select
\DeclareAcronym{AM}{
	short = AM,
	long  = Amplitud Modulada}
%     * AMLCD: Active-matrix liquid crystal display
\DeclareAcronym{ANAC}{
	short = ANAC,
	long  = Agencia Nacional de Aviaci\'on Civil
}

%     * ANC: Active noise cancellation
%     * ANN: Annunciator – caution warning system normally containing visual and
%       audio alerts to the pilot
%     * ANR: Active noise reduction
%     * ANT: Antenna
%     * A/P: Autopilot
%     * APC: Autopilot computer
%     * APS: Autopilot system
%     * APU: Auxiliary power unit
% \acrodef{ARINC}{Aeronautical Radio INCorporated}
% 	ARINC: Aeronautical Radio INCorporated 
\DeclareAcronym{ARINC}{
  short = ARINC ,
  long  = Aeronautical Radio INCorporated
}

%     * ASD: Aircraft situation display
%     * ASDL: Aeronautical satellite data link
\DeclareAcronym{ASI}{
	short = ASI,
	long  = Air Speed Indicator}

\DeclareAcronym{ASK}{
	short = ASK,
	long  = { Modulación Digital por cambio de Amplitud, Amplitud Shift Keying} 
}

%     * ASR: Airport surveillance radar
%     * ASU: Avionics switching unit
%\acrodef{ATC}{Air traffic control} ATC: Air traffic control

\DeclareAcronym{ATC}{
  short = ATC ,
  long  = Air Traffic Control,
  extra = {El Control del Tráfico Aéreo, es un servicio proporcionado por controladores situados en tierra, que guían a las aeronaves en los espacios aéreos controlados y ofrecen información y apoyo a los pilotos en los Espacios Aéreos No Controlados. Tiene como objetivos  proporcionar seguridad, orden y eficiencia al tráfico aéreo. }
}

%     * ATCRBS: Air Traffic Control Radar Beacon System
%     * ATCSS: Air traffic control signaling system
%     * ATI: Unit of measure for instrument size, a standard 3¨û cutout is a 3ATI
\DeclareAcronym{ATM}{
	short = ATM ,
	long = {Gestión de Tráfico Aéreo, Air Traffic Management},
	extra = { El Sistema ATM consiste en la provisión de servicios que  incluyan el espacio aéreo, los aeródromos, las aeronaves, la infraestructura tecnológica y los recursos humanos.}
}

\DeclareAcronym{ATN}{
	short = ATN,
	long  = { Aeronautical Telecomunications Network },
}

\DeclareAcronym{ATSU}{
short = ATSU, 
long = Air Traffic Services Unit}
% 	ATSU: Air Traffic Services Unit%, implementada por AIRBUS en sus aviones (no en los A300/310), 
%		realiza las funciones generalmente usuales en unidades de gesti\'on de comunicaciones
%		de otro tipo de aviones.

%     * Avionics: Aviation + electronics
%     * AWG: American wire gauge


% ***** [edit] B *****
%     * B RNAV: Basic area navigation
%     * BARO: Barometric indication, setting or pressure
%     * BCRS: Back course
%     * BDI: Bearing distance indicator
%     * BGAN: Broadcast Global Area Network
\DeclareAcronym{Bit}{
  short = Bit ,
  long  = BInary Digit,
  extra = {Es una expresión inglesa que significa ``dígito binario''. El concepto se utiliza en la informática para nombrar a una unidad de medida de información que equivale a la selección entre dos alternativas que tienen el mismo grado de probabilidad. Un bit puede tener valor cero o uno.}
}


% ***** [edit] C *****
%     * CAI: Caution annunciator indicator
%     * CAT I: Operational performance Category 1
%     * CAT I Enhanced Allows for lower minimums than CAT I in some cases to CAT
%       2 minimums
%     * CAT II: Operational performance Category II
%     * CAT IIIa: Operational performance Category IIIa
%     * CAT IIIb: Operational performance Category IIIb
%     * CAT IIIc: Operational performance Category IIIc

\DeclareAcronym{CAMU}{
	short = CAMU,
	long  = Communications and Audio Management Unit
}

%     * CODEC: Coder/decoder
%     * CDI: Course deviation indicator
%     * CDTI: Cockpit display of traffic information

\DeclareAcronym{CFDS}{
short = CFDS, 
long = Centralised Fault Display System}
% 	CFDS: Centralised Fault Display System

%     * CFIT: Controlled flight into terrain

\DeclareAcronym{CFR}{
short = CFR, 
long = Code of Federal Regulations
}

\DeclareAcronym{CDFA}{
short = CDFA,
long = Continuous Descent Final Approach, 
extra = {OACI.
Doc 8168. 
Procedures for air navigation services.
Aircraft Operations.
Volume I - Flight Procedures
Sixth Edition, 2018.
A technique, consistent with stabilized approach procedures, for flying
the final approach segment of a non-precision instrument approach procedure as a continuous descent, without
level-off, from an altitude/height at or above the final approach fix altitude/height to a point approximately 15 m
(50 ft) above the landing runway threshold or the point where the flare manoeuvre should begin for the type of
aircraft flown.}
}



\DeclareAcronym{CDU}{
    short = CDU, 
    long = Control Display Unit
    }   
% 	CDU: Control Display Unit

%     * COMM or COM: Communications receiver
\DeclareAcronym{COMM}{
	short = COMM,
	long  = CoMMunications receiver
}





\DeclareAcronym{COTS}{
	short = COTS,
	long  = Commercial-Off-The-Shelf
}


%     * CNS: Communication, navigation, surveillance
%     * CNS/ATM: Communication, navigation, surveillance/air traffic Management
%       [1]
%     * CPDLC: Controller–pilot data link communications
%     * CPS: Cycles per second
%     * CRT: Cathode ray tube
\DeclareAcronym{CRT}{
	short = CRT,
	long  = Cathode Ray Tube
}

%     * CTAF: Common traffic advisory frequency
%     * CV/DFDR: Cockpit voice and digital flight data recorder
%     * CVR: Cockpit voice recorder
%     * CWS: Control wheel steering

% ***** [edit] D *****
%     * DA: Drift angle
%     * DAPs: Downlink of aircraft parameters
%     * DCDU: Data link control and display unit
%     * DG: Directional gyroscope
\DeclareAcronym{DG}{
	short = DG,
	long  = Directional Gyroscope}

%     * DGPS: Differential global positioning system
%     * DH: Decision height

% \acrodef{DITS}{Digital Information Transfer System}
\DeclareAcronym{DITS}{
  short = DITS ,
  long  = Digital Information Transfer System
%  extra = {El Control del Tráfico Aéreo, es un servicio proporcionado por controladores situados en tierra, que guían a las aeronaves en los espacios aéreos controlados y ofrecen información y apoyo a los pilotos en los Espacios Aéreos No Controlados. Tiene como objetivos  proporcionar seguridad, orden y eficiencia al tráfico aéreo. }
}

% 	DITS: Digital Information Transfer System
%     * DLR: Data link recorder
%     * DME: Distance measuring equipment
\DeclareAcronym{DME}{
	short = DME,
	long  = Display Measuring Equipment
}

\DeclareAcronym{DMC}{
	short = DMC,
	long  = Display Management Computers
}
%     * DNC: Direct noise canceling
%     * DP: Departure procedures
\DeclareAcronym{DPCM}{
	short = DPCM,
	long  = {Modulación de Pulsos Codificados Diferenciales, Differential Pulse Code Modulation}
}

%     * DSP: Digital signal processing
%     * DUAT: Direct user access terminal

\DeclareAcronym{DVI}{
	short = DVI ,
	long = Direct Voice Input
}

% ***** [edit] E *****
%     * EADI: Electronic attitude director indicator
\DeclareAcronym{ECAM}{
	short = ECAM, 
	long  = Electronic Centralized Aircraft Monitor
}

% \acrodef{EFD}{Electronic flight display}
\DeclareAcronym{EFD}{
	short = EFD,
	long  = Electronic Flight Display
}

%     * EFIS: Electronic flight instrument system
\DeclareAcronym{EFIS}{
	short = EFIS,
	long  = Electronic Flight Instrument System
}

\DeclareAcronym{EFVS}{
	short = EFVS,
	long  = Enhanced Flight Vision Systems}


%     * EGPWS: Enhanced ground proximity warning system
%     * EGT: Exhaust gas temperature
\DeclareAcronym{EGT}{
	short = EGT,
	long  = Exhaust Gas Temperature}

%     * EHS: Enhanced surveillance
%     * EHSI: Electronic horizontal situation indicator
%       EICAS: Engine indication crew alerting system
\DeclareAcronym{EICAS}{
	short = EICAS,
	long  = Engine Indication Crew Alerting System
}

\DeclareAcronym{EIS}{
	short = EIS,
	long  = Electronic Instrument System
}

%     * ELT: Emergency locator transmitter
%     * ENC: Electronic noise canceling
%     * ENG: Engine
%     * ENR: Electronic noise reduction
%     * EPR: Engine pressure ratio
%     * ETOP: Extended-range twin-engine operation
\DeclareAcronym{ETOP}{
	short = ETOP,
	long  = Extended-Range Twin-Engine Operation
}

\DeclareAcronym{EWD}{
	short = EWD,
	long  = Engine Warning Display}

% ***** [edit] F *****
%     * FADEC: Full authority digital engine control
%     * FANS: Future Air Navigation System
%     * FAT: Free air temperature
%     * FDPS: Flight plan Data Processing System
%     * FDRS: Flight data recorder system
\DeclareAcronym{FDS}{
	short = FDS,
	long  = Flight Director System }

%     * FDU: Flux detector unit
%     * FF: Fuel flow
%FIR: Flight Information Region
\DeclareAcronym{FIR}{
	short = FIR,
	long  = Flight Information Region 
}

%     * FIS-B: Flight information services – broadcast
%     * FLIR: Forward-looking infra-red
%     * FLTA: Forward-looking terrain avoidance

\DeclareAcronym{FM}{
	short = FM,
	long  = Frecuencia Modulada}

% \acrodef{FMC}{Flight Management Computer}
% 	FMC: Flight Management Computer
\DeclareAcronym{FMC}{
	short = FMC,
	long  = Flight Management Computer
}

% \acrodef{FMGS}{Flight Management Guidance System}
% 	FMGS: Flight Management Guidance System
\DeclareAcronym{FMGS}{
	short = FMGS,
	long  = Flight Management Guidance System
}

%       FMS: Flight Management System
\DeclareAcronym{FMS}{
	short = FMS,
	long  = Flight Management System
}
%     * FREQ: Frequency
%     * FSS: Flight service station
\DeclareAcronym{FWC}{
	short = FWC,
	long  = Flight Warning Computer
}

\DeclareAcronym{FSK}{
	short = FSK,
	long  = {Modulación Digital por cambio de Frecuencia, Frecuency Shift Keying} }

%     * FWS: Flight warning system
%     * FYDS: Flight director/ Yaw damper system

% ***** [edit] G *****
\DeclareAcronym{GBAS}{
short = GBAS,
long = Ground based augmentation system,
extra = {\href{https://www.skybrary.aero/sites/default/files/bookshelf/3844.pdf}{GBAS quick facts.}}
}

%     * GCAS: Ground collision avoidance system
%     * GCU: Generator control unit
%     * GDOP: Geometric dilution of precision
%     * GGS: Global positioning system ground station
%     * GHz: Gigahertz
%     * GLNS: GPS Landing and Navigation System
%     * GLNU: GPS landing and navigation unit

\DeclareAcronym{GLONASS}{
	short = GLONASS,
	long  = Global NAvigation Satellite System
}

\DeclareAcronym{GLS}{
short = GLS,
long =  GBAS Landing System
}
%     * GLU: GPS landing unit

% 	GND: Ground (tierra)


    % * GNSS: Global Navigation Satellite System
\DeclareAcronym{GNSS}{
	short = GNSS,
	long  = Global Navigation Satellite System}

    % * GMT: Greenwich Mean Time

%     GPS: Global Positioning Satellite or Global Positioning System
\DeclareAcronym{GPS}{
	short = GPS,
	long  = Global Positioning Satellite o Global Positioning System
}

    % * GPWC: Ground proximity warning computer
%      GPWS: Ground proximity warning system
\DeclareAcronym{GRI}{
	short = GRI,
	long  = Group Repetition Interval.
}


% ***** [edit] H *****
%     * HDG: Heading
%     * HDG SEL: Heading select
%     * HDOP: Horizontal dilution of precision
%     * HF: High frequency
\DeclareAcronym{HF}{
	short = HF,
	long  = High Frequency
}

%     * HHLD: Heading hold
\DeclareAcronym{HI}{
	short = HI,
	long  = Heading Indicator}
%     * HSD: High-speed data
%     * HSI: Horizontal situation indicator
\DeclareAcronym{HSI}{
        short = HSI,
        long  = Horizontal Situation Indicator
}

%     * HSL: Heading select
%     * HUD: Head-up display
\DeclareAcronym{HUD}{
	short = HUD,
	long  = Head-Up Display
}
%     * HMD: Helmet-mounted display
\DeclareAcronym{HMD}{
	short = HMD,
	long  = Helmet-Mounted Display}

% ***** [edit] I *****
\DeclareAcronym{IAC}{
	short = IAC,
	long  = Instrumental Approach Charts}
%IAC: Instrumental Approach Charts


%     * IAS: Indicated airspeed
%     * ID: Identify/Identification or identifier
%     * IDENT: Identify/identifier
%     * IDS: Information display system or integrated display system
%     * IFE: In-flight entertainment
%     * IFR: Instrument flight Rules
\DeclareAcronym{IFR}{
	short = IFR,
	long  = Instrument Flight Rules
}

%     * ILS: Instrument landing system
\DeclareAcronym{ILS}{
	short = ILS,
	long  = Instrument Landing System
}

\DeclareAcronym{IMA}{
	short = IMA,
	long  = Integrated Modular Avionics,
	extra = {
El concepto de aviónica modular integrada intenta reducir la variedad de LRU diseñando la aviónica con componentes comunes. Por ejemplo un tipo de fuente de alimentación, un tipo de unidad de memoria, un tipo de placa de procesamiento y solo algunas variaciones de módulos de entrada / salida. Esto reduce la cantidad de piezas de repuesto necesarias que se deben mantener en stock. Las LRU de IMA tienen funciones más genéricas que en un sistema de aviónica tradicional.
%The Integrated Modular Avionics concept attempts to reduce the variety of LRU's by designing the avionics with common components. E.g. one type of power supply, one type of memory unit, one type of processing board and only a few variations of input/output modules. This reduces the number of replacement parts needed to be kept on stock. The LRUs of IMA have more generic functions than in a traditional avionics system.
}
}

\DeclareAcronym{IMU}{
	short = IMU,
	long  = Inertial Measurement Unit
}

%     * IMC: Instrument meteorological conditions
\DeclareAcronym{IMC}{
	short = IMC,
	long  = Instrument Meteorological Conditions}

%     * InHg: Inch of Mercury
%     * IND: Indicator[disambiguation needed]
%     * INS: Inertial Navigation System
\DeclareAcronym{INS}{
	short = INS,
	long  = Inertial Navigation System
}


%     * ISA: International Standard Atmosphere
%     * ISP: Integrated switching panel
%     * ITT: Interstage turbine temperature
%     * IVSI: Instantaneous vertical speed indicator

% ***** [edit] J *****
%     * JTIDS: Joint Tactical Information Distribution System

% ***** [edit] L *****
%     * LAAS: Local Area Augmentation System
%     * LADGPS: Local Area Differential GPS
%     * LCD: Liquid crystal display
\DeclareAcronym{LCD}{
	short = LCD,
	long  = Liquid Crystal Display
}

%     * LDGPS: Local area differential global positioning satellite
%     * LED: Light-emitting diode
\DeclareAcronym{LED}{
	short = LED,
	long  = Light-Emitting Diode
}

%     * LMM: Locator middle marker
%     * LOC: Localizer
%     * LOM: Locator outer marker
%     * LORAN: Long-range navigation
\DeclareAcronym{LORAN}{
	short = LORAN,
	long  = Long-Range Navigation
}

\DeclareAcronym{LOS}{
  short = LOS ,
  long  = Line-Of-Sight 
}


%     * LRU: Line-replaceable unit
\DeclareAcronym{LRU}{
	short = LRU,
	long  = Line Replaceable Unit,
	extra = { Una LRU es una pieza de hardware que se puede cambiar por una pieza de repuesto en un tiempo relativamente corto abriendo y cerrando sujetadores y conectores. El término LRU se emplea en aviónica pero también en hardware ATC. Ejemplos de esto son FMC, transpondedor, etc.
Cuando se tiene un sistema de aviónica complejo se suele terminar con una gran variedad de LRUs, las cuales tienen una función muy específica. Como desventaja, para poder reemplazar rápidamente las piezas defectuosas, el personal de mantenimiento necesita un gran stock de repuestos.
}
}

\DeclareAcronym{LSB}{
	short = LSB,
	long  = Least Significant Bit
}

\DeclareAcronym{LUF}{
  short = LUF ,
  long  = Lowest Usable Frecuency 
}

\DeclareAcronym{LVT}{
  short = LVT ,
  long  = Low Volume Terminal 
}


% ***** [edit] M *****
%     * MAP: Manifold absolute pressure or missed approach point
%     * MB: Marker beacon
%     * MCBF: Mean cycles between failures

\DeclareAcronym{MCDU}{
short = MCDU, 
long = Multi Control Display Unit}
% 	MCDU: Multi Control Display Unit
%     * MDA: Minimum decent altitude
%     * MEL: Minimum equipment list

\DeclareAcronym{MEMS}{
	short = MEMS,
	long  = MicroElectroMechanical Systems}

%     * MF: Medium frequency
\DeclareAcronym{MF}{
	short = MF,
	long = {Frecuencias Medias, Medium Frequency}
}

%     * MFD: Multi-function display
\DeclareAcronym{MFD}{
	short = MFD,
	long  = Multi-Function Display
}

%     * MFDS: Multi-function display system
%     * MIC: Microphone
%     * MIDS: Multifunctional Information Distribution System
\DeclareAcronym{MIDS}{
  short = MIDS ,
  long  = Multifunctional Information Distribution System
}


%     * MILSPEC: Military specification
\DeclareAcronym{MIU}{
  short = MIU ,
  long  = Mids Interface Unit
}

%     * MKR: Marker beacon
%     * MLS: Microwave landing system
%     * MM: Middle marker
%     * MNPS: [Minimmum navigation performance specifications]
%     * MMD: Moving map display
%     * MOA: Military operations area
%     * Mode A: Transponder pulse-code reporting
%     * Mode C: Transponder code and altitude reporting
%     * Mode S: Transponder code, altitude, and TCAS reporting
%     * MOSArt: Modular Open System Architecture

\DeclareAcronym{MSB}{
	short = MSB,
	long  = Most Significant Bit
}


%     * MSG: Message
\DeclareAcronym{MSL}{
	short = MSL,
	long  = Mean Sea Level}

%     * MSP: Modes S-Specific Protocol
%     * MSSS: Mode S-Specific Services
%     * MTBF: Mean time between failures
%     * MTTF: Mean time to failure

\DeclareAcronym{MUF}{
  short = MUF ,
  long  = Maximum Usable Frecuency 
}


%     * MVFR: Marginal visual flight rules
% ***** [edit] N *****
\DeclareAcronym{NAM}{
	short = NAM,
	long  = Noise Alleviation Measures
}

\DeclareAcronym{NAP}{
short =NAP,
long = Non-Precision Approach
}

%     * NAS: National Airspace System
%     * NAV: Navigation receiver
%     * Navaid: Navigational aid
%     * NAVCOMM: Navigation and communications equipment or receiver
%     * NAVSTAR-GPS: The formal name for the space-borne or satellite navigation
%       system
%     * NCATT: National Center for Aircraft Technician Training
%     * ND: Navigation display
\DeclareAcronym{ND}{
	short = ND,
	long  = Navigation Display
}

%     * NDB: Non-directional radio beacon
\DeclareAcronym{NDB}{
	short = NDB,
	long  = {Baliza No Direccional, Non-Directional Beacon}
}

%     * NFF: No fault found
%     * NM or NMI: Nautical mile
%     * NoTAM: Notice to airmen

\DeclareAcronym{NPA}{
       short= NPA ,
       long = Non-precision approach
}
%     * NVD: Night vision device
%     * NVG: Night vision goggles

% ***** [edit] O *****
%     * OAT: Outside air temperature
%     * OBS: Omnibearing selector
%     * OM: Outer marker
% 	OLED: Organic Light-Emitting Diode
\DeclareAcronym{OLED}{
	short = OLED,
	long  = Organic Light-Emitting Diode
}

% ***** [edit] P *****
%     * PA: Public address system
%     * P-Code: GPS precision code
%     * PAPI: Precision approach path indicator
\DeclareAcronym{PAM}{
	short = PAM,
	long  = {Modulación de Pulsos en Amplitud, Pulse-Amplitude Modulation }
}

\DeclareAcronym{PAR}{
	short = PAR,
	long  = {Radar de Aproximación de Precisión, Precision Approach Radar }
}


\DeclareAcronym{PBN}{
        short = PBN ,
        long  = Performance Based Navigation
}

\DeclareAcronym{PAPI}{
	short = PAPI,
	long  = Precision Approach Path Indicator
}

\DeclareAcronym{PCM}{
	short = PCM,
	long  = {Modulación de Pulsos Codificados, Pulse Code Modulation}
}

%     * PD: Profile descent
%     * PDOP: Position dilution of precision
\DeclareAcronym{PDS}{
	short = PDS,
	long  = Portable Data Store
}


%     * PFD: Primary flight display or primary flight director
\DeclareAcronym{PFD}{
	short = PFD,
	long  = Primary Flight Display
}

\DeclareAcronym{PM}{
	short = PM,
	long  = {Fase Modulada, Phase Modulation}
}

%     * PMG: Permanent magnet generator
%     * PND: Primary navigation display
%     * PNR: Passive noise reduction
%     * POS: Position[disambiguation needed]
\DeclareAcronym{PPM}{
	short = PPM,
	long  = {Modulación de Pulsos de Posici\'on, Pulse Position Modulation}
}

\DeclareAcronym{PSK}{
	short = PSK,
	long  = {Modulación Digital por cambio de Fase, Phase Shift Keying}
}

\DeclareAcronym{PWM}{
	short = PWM,
	long = {Modulación por Ancho de Pulso, Pulse Wide Modulation}
}

%     * P-RNAV: precision area navigation
%     * PSR: Primary surveillance radar
%     * PTT: Push-to-talk

\DeclareAcronym{PSR}{
	short = PSR,
	long  = {Radar Primario de Vigilancia, Primary Surveillance Radar }
}



% ***** [edit] R *****
%     * RA: Resolution advisory (TCAS)
\DeclareAcronym{RAAC}{
	short = RAAC,
	long  = Regulaciones Argentinas de Aviaci\'on Civil
}

\DeclareAcronym{RAF}{
	short = RAF,
	long  = Royal Air Force}

%     * RAI: Radio altimeter indicator
%     * RAIM: Receiver-autonomous integrity monitoring, also remote autonomous
%       integrity monitoring
%     * RALT: Radar or radio altimeter
%     * RAT: Ram air turbine
\DeclareAcronym{RBI}{
	short = RBI,
	long  = Relative Bearing Indicator
}

%     * RCR: Reverse current relay
%     * RCVR: Receiver
%     * RDMI: Radio distance magnetic indicator
%     * RDP: Radar data processing system
%     * RDR: Radar
%     * REF: Reference
%     * REIL: Runway end identifier lights
%     * REL: Relative[disambiguation needed]
%     * RF: Radio frequency
\DeclareAcronym{RF}{
	short = RF,
	long  = Radio Frequency }


%     * RFI: Radio frequency interference
%     * RHSM: Reduced horizontal separation minimal
%     * RLG: Ring laser gyroscope
%     * RLY: Relay
%     * RMI: Radio magnetic indicator
\DeclareAcronym{RMI}{
	short = RMI,
	long  = Radio Magnetic Indicator
}

%     * R-NAV: Area navigation
%     * RNG: Range

\DeclareAcronym{RNP}{
        short = RNP,
        long = Required navigation performance
        }

%     * ROC: Rate of climb
%     * ROD: Rate of descent
%     * RPA: Remotely piloted aircraft (Unmanned aerial vehicle)
%     * RPM: Revolutions per minute
%     * RTE: Route
%     * RVR: Runway visual range
%     * RVSM: Reduced vertical separation minimum
%     * RX: Receiver

% ***** [edit] S *****
\DeclareAcronym{SBAS}{
short = SBAS,
long = Satellite Based Augmentation System,
extra = {Sistema de Aumentación Basado en Satélites, es un sistema de corrección de las señales que los Sistemas Globales de Navegación por Satélite (GNSS) transmiten al receptor GPS del usuario. Los sistemas SBAS mejoran el posicionamiento horizontal y vertical del receptor y dan información sobre la calidad de las señales. Aunque inicialmente fue desarrollado para dar una precisión mayor a la navegación aérea, cada vez se está generalizando más su uso en otro tipo de actividades que requieren de un uso sensible de la señal GPS. }
}


\DeclareAcronym{SD}{
	short = SD,
	long  = System Display
}

\DeclareAcronym{SDAC}{
	short = SDAC,
	long  = System Data Acquisition Concentrator
}
%     * SAT: Static air temperature
%     * SATCOM: Satellite communication
%     * SATNAV: Satellite navigation
%     * SD: Secure digital
%     * SELCAL: Selective calling

\DeclareAcronym{SID}{
short = SID,
long = Standard Instrument Departure
}
%       SID: Standard Instrument Departure

%     * SIU: Satellite interface unit
%     * S: Sensitivity Level
%     * SMS: Short Messaging Service
%     * SNR: Signal-to-noise ratio
%     * SPKR: Speaker
%     * SQ or SQL: Squelch
%     * SSCV/DR: Solid-state cockpit voice/data recorder
%     * SSCVR: Solid-state cockpit voice recorder
%     * SSFDR: Solid-state flight data recorder

\DeclareAcronym{SSM}{
  short = SSM ,
  long  = Sign/Status Matrix
}

%     * SSR: Secondary surveillance radar

\DeclareAcronym{STAR}{
short = STAR, 
long = Standard Terminal Arrival Route}
%       STAR: Standard Terminal Arrival Route

% \acrodef{STARS}{Standard Terminal Automation Replacement System}
\DeclareAcronym{STARS}{
short = STARS,  
long =  Standard Terminal Automation Replacement System
}



%     * STC: Supplemental Type Certificate
%     * STCA: Short-Term Conflict Alert
%     * STP: Standard temperature and pressure
\DeclareAcronym{STP}{
	short = STP,
	long = Standard Temperature and Pressure
}

%     * SUA: Special use airspace

% ***** [edit] T *****
%     * TA: Traffic advisory (see TCAS)
%     * TACAN: Tactical air navigation system
\DeclareAcronym{TACAN}{
	short = TACAN,
	long = Tactical Air Navigation System
}

%     * Tach: Tachometer
%     * TAD: Terrain awareness display
%     * TAF: Terminal area forecast
%     * TAS: True airspeed
%     * TAT: True air temperature, or total air temperature
%     * TAWS: Terrain awareness warning system
\DeclareAcronym{TAWS}{
	short = TAWS,
	long  = Terrain Awareness Warning System
}

%     * TBO: Time before overhaul, or time between overhaul
\DeclareAcronym{TC}{
	short = TC,
	long  = Turn Coordinator}

%     * TCA: Throttle control assembly, or terminal control area
%     * TCAS: Traffic collision avoidance system
%     * TCF: Terrain clearance floor
%     * TCN: TACAN
%     * TCU: TACAN control unit
%     * TDOP: Time dilution of precision
%     * TDR: Transponder (in some cases)
%     * TERPS]]: Terminal instrument procedures, or terminal enroute procedures
%     * TFR: Temporary flight restrictions
%     * TFT: Thin-film transistor
%     * TGT: Turbine gas temperature, or target
%     * THDG: True heading
%     * TIAS: True indicated airspeed
%     * TIS: Traffic information service
%     * TK: Track angle
%     * TKE: Track-angle error
%     * TLA: Three-letter acronym
%     * TOGA: Takeoff/Go-around switch, Takeoff/go-around thrust
%     * TOT: Turbine outlet temperature
%     * TR or T/R: Transmitter receiver or transceiver
%     * TRACON: Terminal radar approach control
%     * TRANS: Transmit, Transmission, or Transition[disambiguation needed]
%     * TRK: Track
%     * TRP: Mode S transponder
%     * TTR: TCAS II transmitter/receiver
%     * TTS: Time to station
%     * TVE: Total vertical error
%     * TWDL: Two-way data link, or terminal weather data link
%     * TWDR: Terminal Doppler Weather Radar
%     * TWIP: Terminal weather information for pilots
%     * TWR: Terminal weather radar
%     * TX: Transmit

% ***** [edit] U *****
%     * UART: Universal asynchronous receiver transmitter
%     * UAV: Unmanned aerial vehicle
%     * UHF: Ultra-high frequency
\DeclareAcronym{UHF}{
	short = UHF,
	long  = Ultra-High Frequency
}

%     * ULB: Underwater locator beacon
%     * USB: Universal Serial Bus
%     * UTC: Universal Time Coordinate
\DeclareAcronym{UTC}{
	short = UTC,
	long  = Universal Time Coordinate}

\DeclareAcronym{UTM}{
	short = UTM,
	long  = Universal Transverse Mercator }

% ***** [edit] V *****
%     * V: Volts or voltage
%     * VASI: Visual approach slope indicator
\DeclareAcronym{VASI}{
	short = VASI,
	long  = Visual Approach Slope Indicator 
}

%     * VDL: VHF Data Link
%     * VDR: VHF digital radio
%     * VFO: Variable frequency oscillator
%     * VFR: Visual flight rules
\DeclareAcronym{VFR}{
	short = VFR,
	long  = Visual Flight Rules
}

%     * VG/DG: Vertical gyroscope/directional gyroscope
%     * VGA: Video Graphics Array
%     * VHF: Very high frequency
\DeclareAcronym{VHF}{
  short = VHF ,
  long  = Very High Frecuency
}

\DeclareAcronym{VNAV}{
short = VNAV,
long = Vertical NAVigation
}



%     * V/L: VOR/Localizer
%     * VMC: Visual meteorological conditions or minimum control speed with
%       critical engine out

%     * VNE: Never exceed speed
%     * VNO: Maximum structural cruising speed
%     * VNR: VHF navigation receiver

\DeclareAcronym{VOGAD}{
	short = VOGAD,
	long  = Voice Operated Gain Adjustable Device
}

%     * VOR: VHF omnidirectional range and ranging
\DeclareAcronym{VOR}{
	short = VOR,
	long  = VHF Omnidirectional Range and Ranging }

%     * VOR/DME: VOR with Distance measuring equipment
%     * VOR/MB: VOR marker beacon
%     * VORTAC: VOR and TACAN combination

\DeclareAcronym{VOS}{
	short = VOS,
	long  = Voice Operated Switch
}

%     * VOX: Voice transmission
%     * VPATH: Vertical path
%     * V/R: Voltage regulator
%     * V/REF: Reference velocity
%     * V/S: Vertical speed
%     * VSI: Vertical speed indicator
\DeclareAcronym{VSI}{
	short = VSI,
	long  = Vertical Speed Indicator}

%     * VSM: Vertical separation limit
%     * VSO: Stall speed in landing configuration
%     * VSWR: Voltage–standing wave ratio
%     * V/TRK: Vertical track
%     * VX: Speed for best angle of climb
%     * VY: Speed for best rate of climb

% ***** [edit] W *****
%     * WAAS: Wide Area Augmentation System
%     * WD/WINDR: Wind direction
%     * WMA: WXR waveguide adapter
%     * WMI: WXR indicator mount
%     * WMS: Wide-area master station
%     * WMSC: Weather message switching center
%     * WMSCR: Weather message switching center replacement
%     * WPT: Waypoint
%     * WRT: WXR receiver transmitter
%     * WX: Weather
%     * WXR: Weather radar system
%     * WYPT: Waypoint

% ***** [edit] X *****
%     * XCVR: Transceiver
%     * XFR: Transfer
%     * XMIT: Transmit
%     * XMSN: Transmission
%     * XMTR: Transmitter
%     * XPDR: Transponder
%     * XTK: Crosstrack

% ***** [edit] Y *****
%     * YD: Yaw damper

% ***** [edit] See also *****
%     * Avionics
%     * List of aviation, aerospace and aeronautical abbreviations

% ***** [edit] References *****
%    1. ^ http://wwwicaoint/icao/en/ro/rio/execsumpdf


% Retrieved from "http://en.wikipedia.org/w/
% index.php?title=Acronyms and abbreviations in avionics&amp;oldid=518531528"
